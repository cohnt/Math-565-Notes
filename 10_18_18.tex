\documentclass[10pt,letterpaper]{article}
\usepackage[utf8]{inputenc}
\usepackage{amsmath}
\usepackage{amsfonts}
\usepackage{amssymb}
\usepackage{ragged2e}
\usepackage[letterpaper, margin=1in]{geometry}
\usepackage{graphicx}
\usepackage{cancel}
\usepackage{mathtools}
\usepackage{tabularx}
\usepackage{arydshln}
\usepackage{tensor}
\usepackage{array}
\usepackage{xcolor}
\usepackage[boxed]{algorithm}
\usepackage[noend]{algpseudocode}
\usepackage{listings}
\usepackage{textcomp}
\usepackage[pdf,tmpdir,singlefile]{graphviz}

%%%%%%%%%%%%%%%%%%%%%%%%%%%%%
% Formatting commands
%%%%%%%%%%%%%%%%%%%%%%%%%%%%%
\newcommand{\n}{\hfill\break}
\newcommand{\lemma}[1]{\par\noindent\settowidth{\hangindent}{\textbf{Lemma: }}\textbf{Lemma: }#1}
\newcommand{\defn}[1]{\par\noindent\settowidth{\hangindent}{\textbf{Defn: }}\textbf{Defn: }#1\n}
\newcommand{\thm}[1]{\par\noindent\settowidth{\hangindent}{\textbf{Thm: }}\textbf{Thm: }#1\n}
\newcommand{\prop}[1]{\par\noindent\settowidth{\hangindent}{\textbf{Prop: }}\textbf{Prop: }#1\n}
\newcommand{\cor}[1]{\par\noindent\settowidth{\hangindent}{\textbf{Cor: }}\textbf{Cor: }#1\n}
\newcommand{\ex}[1]{\par\noindent\settowidth{\hangindent}{\textbf{Ex: }}\textbf{Ex: }#1\n}
\newcommand{\proven}{\;$\square$\n}
\newcommand{\problem}[1]{\par\noindent{#1}\n}
\newcommand{\problempart}[2]{\par\noindent\indent{}\settowidth{\hangindent}{\textbf{(#1)} \indent{}}\textbf{(#1)} #2\n}
\newcommand{\ptxt}[1]{\textrm{\textnormal{#1}}}
\newcommand{\inlineeq}[1]{\centerline{$\displaystyle #1$}}
\newcommand{\pageline}{\noindent\rule{\textwidth}{0.1pt}}

%%%%%%%%%%%%%%%%%%%%%%%%%%%%%
% Math commands
%%%%%%%%%%%%%%%%%%%%%%%%%%%%%
% Set Theory
\newcommand{\card}[1]{\left|#1\right|}
\newcommand{\set}[1]{\left\{#1\right\}}
\newcommand{\ps}[1]{\mathcal{P}\left(#1\right)}
\newcommand{\pfinite}[1]{\mathcal{P}^{\ptxt{finite}}\left(#1\right)}
\newcommand{\naturals}{\mathbb{N}}
\newcommand{\N}{\naturals}
\newcommand{\integers}{\mathbb{Z}}
\newcommand{\Z}{\integers}
\newcommand{\rationals}{\mathbb{Q}}
\newcommand{\Q}{\rationals}
\newcommand{\reals}{\mathbb{R}}
\newcommand{\R}{\reals}
\newcommand{\complex}{\mathbb{C}}
\newcommand{\C}{\complex}
\newcommand{\comp}{^{\complement}}
\newcommand{\Hom}{\ptxt{Hom}\>}

% Graph Theory
\renewcommand{\deg}[1]{\ptxt{deg}\left(#1\right)}
\newcommand{\degp}[1]{\ptxt{deg}^{+}\!\!\left(#1\right)}
\newcommand{\degn}[1]{\ptxt{deg}^{-}\!\!\left(#1\right)}
\newcommand{\Prob}{\mathbb{P}}
\newcommand{\Avg}{\mathbb{E}}

% Standard Math
\newcommand{\inv}{^{-1}}
\newcommand{\abs}[1]{\left|#1\right|}
\newcommand{\ceil}[1]{\left\lceil{}#1\right\rceil}
\newcommand{\floor}[1]{\left\lfloor{}#1\right\rfloor{}}
\newcommand{\conj}[1]{\overline{#1}}
\newcommand{\of}{\circ}
\newcommand{\tri}{\triangle}
\newcommand{\inj}{\hookrightarrow}
\newcommand{\surj}{\twoheadrightarrow}
\newcommand{\mapsfrom}{\mathrel{\reflectbox{\ensuremath{\mapsto}}}}
\newcommand{\Graph}{\ptxt{Graph}\>}
\newcommand{\ndiv}{\nmid}
\renewcommand{\epsilon}{\varepsilon}

% Linear Algebra
\newcommand{\Id}{\textrm{\textnormal{Id}}}
\newcommand{\im}{\textrm{\textnormal{im}}}
\newcommand{\norm}[1]{\abs{\abs{#1}}}
\newcommand{\tpose}{^{T}}
\newcommand{\iprod}[1]{\left<#1\right>}
\newcommand{\trace}{\ptxt{tr}}
\newcommand{\chgBasMat}[3]{\!\!\tensor*[_{#1}]{\left[#2\right]}{_{#3}}}
\newcommand{\vecBas}[2]{\tensor*[]{\left[#1\right]}{_{#2}}}
\newcommand{\GL}{\ptxt{GL}\>}
\newcommand{\Mat}{\ptxt{Mat}\>}

% Topology
\newcommand{\closure}[1]{\overline{#1}}
\newcommand{\uball}{\mathcal{U}}
\newcommand{\Int}{\ptxt{Int}\>}
\newcommand{\Ext}{\ptxt{Ext}\>}
\newcommand{\Bd}{\ptxt{Bd}\>}

% Proofs
\newcommand{\st}{s.t.}
\newcommand{\unique}{!}

% Algorithms
\algrenewcommand{\algorithmiccomment}[1]{\hskip 1em \texttt{// #1}}
\algrenewcommand\algorithmicrequire{\textbf{Input:}}
\algrenewcommand\algorithmicensure{\textbf{Output:}}

%%%%%%%%%%%%%%%%%%%%%%%%%%%%%
% Other commands
%%%%%%%%%%%%%%%%%%%%%%%%%%%%%
\newcommand{\flag}[1]{\textbf{\textcolor{red}{#1}}}

%%%%%%%%%%%%%%%%%%%%%%%%%%%%%
% Make l's curvy in math environments
%%%%%%%%%%%%%%%%%%%%%%%%%%%%%
\mathcode`l="8000
\begingroup
\makeatletter
\lccode`\~=`\l
\DeclareMathSymbol{\lsb@l}{\mathalpha}{letters}{`l}
\lowercase{\gdef~{\ifnum\the\mathgroup=\m@ne \ell \else \lsb@l \fi}}%
\endgroup

\author{Thomas Cohn}
\title{Ramsey's Theorem}
\date{10/18/2018} % Can also use \today

\begin{document}
\maketitle
\setlength\RaggedRightParindent{\parindent}
\RaggedRight

\par\noindent Extremal Problems: If $G$ on $n$ vertices avoids the same structure, how big is $\card{E}$?\n

\thm{(Tur\'an) If $G$ on $n$ vertices avoids $K_{k}$, then $\card{E}\le\frac{k-2}{2(k-1)}n^{2}+C$, where $C$ is a constant term.}

\par\noindent If both $G$ and $\overline{G}$ on $n$ vertices avoid $K_{k}$, then how big is $n$?\n

\ex{$k=2$, then $n<2$.}

\prop{If $n=6$, and the edges of $K_{n}$ are colored R/B, then we can find a monochromatic triangle.\n
Proof: Look at vertex $v$. $N(v)=X\sqcup{}Y$, where $X=\set{x:xv\ptxt{ is red}}$ and $Y=\set{y:yv\ptxt{ is blue}}$.\n
$\card{X}+\card{Y}=5$, so one of them (WOLOG $X$) has size at least $3$.\n
\n
Say $x_{1},x_{2},x_{3}\in{}X$. Then $vx_{1}x_{2}$, $vx_{1}x_{3}$, and $vx_{2}x_{3}$ are all triangles. If any $x_{1}x_{2}$, $x_{1}x_{3}$, and $x_{2}x_{3}$ are red, then we have a red triangle $vx_{i}x_{j}$. If all are blue, then $x_{1}x_{2}x_{3}$ is a blue triangle.\proven}

\defn{For $r,s\in\N$, the \underline{Ramsey Number} $R(r,s)$ is the minimum $n\in\N$ \st{} any R/B coloring of the edges in $K_{n}$ contains a red $K_{r}$ or a blue $K_{s}$.}

\ex{$R(3,3)=6$}

\thm{(Ramsey) $R(r,s)<\infty$, $\forall{}r,s\in\N$.\n
Proof: Induction on $r$, $s$.\n
Assume we know $R(r-1,s)$ and $R(r,s-1)$ are finite.\n
Claim: $R(r,s)\le{}R(r-1,s)+R(r,s-1)$. Let $n=R(r-1,s)+R(r,s-1)\in\N$. Consider a R/B coloring of $K_{n}$. Let $X=\set{x:xv\ptxt{ is red}}$ and $Y=\set{y:yv\ptxt{ is blue}}$. Then $\card{X}+\card{Y}=n-1=a+b-1$. Thus $\card{X}\ge{}a$ or $\card{Y}\ge{}b$.\n
\n
Case 1: Let $\card{X}\ge{}a$. Let $G_{x}$ be the complete subgraph on $X$. Then $G_{x}$ has a red $K_{r-1}$ or a blue $K_{s}$. If $G_{x}$ contains a red $K_{r-1}$, then $G$ has a $K_{r}$. If $G_{x}$ contains a blue $K_{s}$, then $G$ has a $K_{s}$.\n
\n
Case 2: Just do the same thing.\n
\n
We conclude that if $n=a+b$, then any R/B coloring of edges in $K_{n}$ contains a red $K_{r}$ or blue $K_{s}$, and $n\ge{}R(r,s)$.\proven}

\ex{$R(4,4)=18$\n
$R(3,5)=14$\n
$R(3,6)=18$\n
$R(3,7)=23$\n
$R(3,8)=28$\n
$R(3,9)=36$\n}

\cor{$R(r,s)\le\binom{r+s-2}{r-1}$. If $r=s$, then $R(r,s)\le\binom{2r-2}{r-1}\le{}2^{2r-2}$.\n
Proof: Assume $R(r-1,s)\le\binom{r+s-3}{r-2}$ and $R(r,s-1)\le\binom{r+s-3}{r-1}$.\n
Then $R(r,s)\le{}R(r-1,s)+R(r,s-1)=\binom{r+s-3}{r-2}+\binom{r+s-3}{r-1}=\binom{r+s-2}{r-1}$.\proven}

\thm{$R(r,r)\ge{}2^{(r-2)/2}$.\n
Proof: $n=2^{(r-2)/2}$. There is a R/B coloring of edges in $K)_{n}$ with \underline{no} red $K_{r}$ or blue $K_{r}$. Consider $X\subset[n]$ with $\card{X}=r$. Then $\Prob(G_{x}\ptxt{ is mono})=\frac{1}{2^{\binom{r}{2}}}+\frac{1}{2^{\binom{r}{2}}}=2^{1-\binom{r}{2}}$\n
Let $Y=\card{\set{X\subset[n]:\card{X}=r\land{}G_{x}\ptxt{ is mono}}}$.\n
Then $\Avg[y]=\sum_{X\subset[n],\card{X}=r}\Prob(G_{x}\ptxt{ is mono})=\binom{n}{r}\cdot{}2^{1-\binom{r}{2}}$, and $\binom{n}{r}<n^{r}$\n
So $\Avg[y]<n^{r}\cdot{}2^{1-\binom{n}{r}}=\left(2^{(r-2)/2}\right)^{r}=2^{1-\frac{r(r-1)}{2}}=2^{r(r-2)/2-r(r-1)/2+1}=2^{-r/2+1}<1$\n
So there exists a set of coin flips for which $y=0$, and therefore, there is a R/B coloring on $K_{n}$ for which no $G_{x}$ is mono.\proven}

\end{document}