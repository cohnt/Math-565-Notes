\documentclass[10pt,letterpaper]{article}
\usepackage[utf8]{inputenc}
\usepackage{amsmath}
\usepackage{amsfonts}
\usepackage{amssymb}
\usepackage{ragged2e}
\usepackage[letterpaper, margin=1in]{geometry}
\usepackage{graphicx}
\usepackage{cancel}
\usepackage{mathtools}
\usepackage{tabularx}
\usepackage{arydshln}
\usepackage{tensor}
\usepackage{array}
\usepackage{xcolor}

% Formatting commands
\newcommand{\n}{\hfill\break}
\newcommand{\lemma}[1]{\par\noindent\settowidth{\hangindent}{\textbf{Lemma: }}\textbf{Lemma: }#1\n}
\newcommand{\defn}[1]{\par\noindent\settowidth{\hangindent}{\textbf{Defn: }}\textbf{Defn: }#1\n}
\newcommand{\thm}[1]{\par\noindent\settowidth{\hangindent}{\textbf{Thm: }}\textbf{Thm: }#1\n}
\newcommand{\proven}{\;$\square$\n}
\newcommand{\problem}[1]{\par\noindent{#1}\n}
\newcommand{\problempart}[2]{\par\settowidth{\hangindent}{\textbf{(#1)} \indent{}}\textbf{(#1)} #2\n}
\newcommand{\ptxt}[1]{\textrm{\textnormal{#1}}}
\newcommand{\inlineeq}[1]{\n\centerline{$\displaystyle #1$}}
\newcommand{\pageline}{\noindent\rule{\textwidth}{0.1pt}\n}

% Math commands
\renewcommand{\deg}[1]{\ptxt{deg}\left(#1\right)}
\newcommand{\card}[1]{\left|#1\right|}
\newcommand{\set}[1]{\left\{#1\right\}}
\newcommand{\inv}{^{-1}}
\newcommand{\abs}[1]{\left|#1\right|}
\newcommand{\st}{s.t.}
\newcommand{\naturals}{\mathbb{N}}
\newcommand{\N}{\naturals}
\newcommand{\integers}{\mathbb{Z}}
\newcommand{\Z}{\integers}
\newcommand{\rationals}{\mathbb{Q}}
\newcommand{\Q}{\rationals}
\newcommand{\reals}{\mathbb{R}}
\newcommand{\R}{\reals}
\newcommand{\complex}{\mathbb{C}}
\newcommand{\C}{\complex}
\newcommand{\ceil}[1]{\left\lceil{}#1\right\rceil}

% Other commands
\newcommand{\flag}[1]{\textbf{\textcolor{red}{#1}}}

\author{Dr. Danny Nguyen\\ \small\textit{Transcribed by Thomas Cohn}}
\title{Trees, Caylees Theorem}
\date{9/4/2018}

\begin{document}
\maketitle
\setlength\RaggedRightParindent{\parindent}
\RaggedRight

\defn{A \underline{graph} is the ordered pair $G=(V,E)$, where $V$ is the set of vertices and $E$ is the set of edges.}

\defn{A graph is said to be \underline{connected} if you cannot write $V=V_{1}\sqcup{}V_{2}$ such that every pair of vertices $v_{1}\in{}V_{1}$ and $v_{2}\in{}V_{2}$ is not adjacent.}

\defn{A \underline{tree} is a connected graph with no cycles.}

\defn{A \underline{forest} is a collection of disjoint trees.}

\defn{The \underline{degree} of a vertex $v$, $\deg{v}$, is the number of edges incident to $v$.}

\thm{$\displaystyle\sum_{v\in{}V}\deg{v}=2\card{E}$\n
The reason for this should be obvious.}

\defn{A vertex of degree $1$ in a tree is called a \underline{leaf}.}

\par\noindent $[n]$ is a set of n labelled vertices. $C(n)$ is defined as the number of distinct trees on $[n]$.\n
For example, $C(2)=1$, $C(3)=3$, and $C(4)=16$. Is there some sort of pattern? Perhaps even a formula?

\par\noindent Note that $C(n+1)$ is the number of rooted forests on $[n]$.

\thm{Cayley's Theorem\n
$\displaystyle{}C(n)=n^{n-2}$}

\defn{A \underline{rooted tree} is a tree on $[n]$ with a distinguished vertex (the \underline{root}).}

\defn{A \underline{rooted forest} is a forest where every tree is a rooted tree.}

\par\noindent $\overrightarrow{C}(n)$ is defined as the number of rooted trees on $[n]$. A tree with $n$ vertexes could be made into $n$ distinct rooted trees, depending on where the root is placed. So if Cayley's Theorem is true, we would expect $\overrightarrow{C}(n)=n\cdot{}n^{n-2}=n^{n-1}$.\n

\par\noindent Observe that if we have rooted forest $\overrightarrow{F}$, and we remove an edge $\overrightarrow{e}$, we get a rooted forest $\overrightarrow{F}'$ that has one more tree than $\overrightarrow{F}$.

\newpage

\par\noindent\textbf{Cayley's Theorem Proof 1: Double Counting}
\par\noindent Let $F_{n,k}=\set{\textrm{rooted $k$-forests on }[n]}$. Thus, $F_{n,1}=\set{\textrm{rooted trees on }[n]}$.
\par\noindent Consider some $F_{1}\in{}F_{n,1}$. We can remove an edge, and call this new rooted forest $F_{2}\in{}F_{n,2}$. We can repeat this process all the way to $F_{n}\in{}F_{n,n}$. This will leave us with $n$ vertices, and no edges connecting any of them; we can see that $\card{F_{n,n}}=1$. We can also see that there are $(n-1)!$ possible ways to remove the edges from any $F_{1}$ to reach $F_{n}$.\n

\par\noindent But how many ways are there to add edges from $F_{n}\in{}F_{n,n}$ up to $F_{1}\in{}F_{n,1}$? We can pick any two vertices in $F_{n}$, and the edge between them could face either direction, so there are $\binom{n}{2}\cdot{}2=n(n-1)$ ways to grow from $F_{n,n}$ to $F_{n,n-1}$. For growing from $F_{n,k}$ to $F_{n,k-1}$, we can choose any vertex, and chain an edge to it from any tree \textit{other than the one it is a part of}. So there are $n(k-1)$ ways to grow from $F_{n,k}$ to $F_{n,k-1}$.\n

\par\noindent Therefore, we have $\displaystyle\prod_{k=n}^{2}n(k-1)=n^{n-1}\cdot{}(n-1)!$ ways to grow from $F_{n,n}$ to $F_{n,1}$. And we have $(n-1)!$ ways to remove the edges from each of the trees in $F_{n,1}$ (sending it back to $F_{n,n}$).
\par\noindent So we must have $\displaystyle\overrightarrow{C}(n)\cdot(n-1)!=n^{n-1}\cdot{}(n-1)!$, and if $\overrightarrow{C}(n)=n^{n-1}$, then $C(n)=n^{n-2}$.\proven

\par\noindent\textbf{Cayley's Theorem Proof 2: Pr\"ufercode}
\par\noindent We will look at a function $f:T\mapsto{}w\in[n]^{n-2}$, where $w$ is obtained via a recursive process:
\begin{enumerate}
\item Select the smallest leaf in $T$, denoted $v$.
\item Look at the neighbor of $v$, denoted $v'$.
\item Put $v'$ into $w$, then delete $v$ from $T$, and return to step 1.
\end{enumerate}
\par\noindent This process ends when there are only $2$ vertices left.\n

\lemma{$v\in{}T$ is a leaf $\leftrightarrow{}v\not\in{}w$\n
Proof: Assume that $v$ is a leaf. Then we know if does not have a child pointing to it, so it could never be inserted into $w$ as per our recursive algorithm. Thus, $v\not\in{}w$.\n
Assume that $v$ is not a leaf. Then we know that $v$ has at least $2$ neighbors. Since the algorithm terminates when there are only $2$ vertices remaining, we know that at least one of the neighbors must be deleted, so $v\in{}w$.\proven}

\par\noindent Now, we must define the inverse function $g=f\inv:w\in{}[n]^{n-2}\mapsto{}T$ in order to obtain a bijection. Given $w=(w_{1},w_{2},\ldots,w_{n-2})$, define $v=\min\set{[n]\setminus\set{w_{1},\ldots,w_{n-2}}}$. Then let $T$ be the graph with vertices $v$ and $w_{1}$ and a single edge connecting them, and let $w'=(w_{2},\ldots,w_{n-2})$. We can then repeat this process recursively on $w'$.

\par\noindent\;

\par\noindent\flag{There's probably more to come on Thursday's lecture.}

\par\noindent\flag{I don't think we ever actually finished this in class.}

\end{document}