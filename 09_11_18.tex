\documentclass[10pt,letterpaper]{article}
\usepackage[utf8]{inputenc}
\usepackage{amsmath}
\usepackage{amsfonts}
\usepackage{amssymb}
\usepackage{ragged2e}
\usepackage[letterpaper, margin=1in]{geometry}
\usepackage{graphicx}
\usepackage{cancel}
\usepackage{mathtools}
\usepackage{tabularx}
\usepackage{arydshln}
\usepackage{tensor}
\usepackage{array}
\usepackage{xcolor}

%%%%%%%%%%%%%%%%%%%%%%%%%%%%%
% Formatting commands
%%%%%%%%%%%%%%%%%%%%%%%%%%%%%
\newcommand{\n}{\hfill\break}
\newcommand{\lemma}[1]{\par\noindent\settowidth{\hangindent}{\textbf{Lemma: }}\textbf{Lemma: }#1\n}
\newcommand{\defn}[1]{\par\noindent\settowidth{\hangindent}{\textbf{Defn: }}\textbf{Defn: }#1\n}
\newcommand{\thm}[1]{\par\noindent\settowidth{\hangindent}{\textbf{Thm: }}\textbf{Thm: }#1\n}
\newcommand{\ex}[1]{\par\noindent\settowidth{\hangindent}{\textbf{Ex: }}\textbf{Ex: }#1\n}
\newcommand{\proven}{\;$\square$\n}
\newcommand{\problem}[1]{\par\noindent{#1}\n}
\newcommand{\problempart}[2]{\par\settowidth{\hangindent}{\textbf{(#1)} \indent{}}\textbf{(#1)} #2\n}
\newcommand{\ptxt}[1]{\textrm{\textnormal{#1}}}
\newcommand{\inlineeq}[1]{\n\centerline{$\displaystyle #1$}}
\newcommand{\pageline}{\noindent\rule{\textwidth}{0.1pt}}

%%%%%%%%%%%%%%%%%%%%%%%%%%%%%
% Math commands
%%%%%%%%%%%%%%%%%%%%%%%%%%%%%
% Set Theory
\newcommand{\card}[1]{\left|#1\right|}
\newcommand{\set}[1]{\left\{#1\right\}}
\newcommand{\ps}[1]{\mathcal{P}(#1)}
\newcommand{\naturals}{\mathbb{N}}
\newcommand{\N}{\naturals}
\newcommand{\integers}{\mathbb{Z}}
\newcommand{\Z}{\integers}
\newcommand{\rationals}{\mathbb{Q}}
\newcommand{\Q}{\rationals}
\newcommand{\reals}{\mathbb{R}}
\newcommand{\R}{\reals}
\newcommand{\complex}{\mathbb{C}}
\newcommand{\C}{\complex}
\newcommand{\comp}{^{\complement}}

% Graph Theory
\renewcommand{\deg}[1]{\ptxt{deg}\left(#1\right)}

% Standard Math
\newcommand{\inv}{^{-1}}
\newcommand{\abs}[1]{\left|#1\right|}
\newcommand{\ceil}[1]{\left\lceil{}#1\right\rceil}
\newcommand{\floor}[1]{\left\lfloor{}#1\right\rfloor{}}
\newcommand{\conj}[1]{\overline{#1}}
\newcommand{\of}{\circ}
\newcommand{\tri}{\triangle}
\newcommand{\inj}{\hookrightarrow}
\newcommand{\surj}{\twoheadrightarrow}
\newcommand{\mapsfrom}{\mathrel{\reflectbox{\ensuremath{\mapsto}}}}

% Linear Algebra
\newcommand{\Id}{\textrm{\textnormal{Id}}}
\newcommand{\im}{\textrm{\textnormal{im}}}
\newcommand{\norm}[1]{\abs{\abs{#1}}}
\newcommand{\tpose}{^{T}}
\newcommand{\iprod}[1]{\left<#1\right>}
\newcommand{\trace}{\ptxt{tr}}
\newcommand{\chgBasMat}[3]{\!\!\tensor*[_{#1}]{\left[#2\right]}{_{#3}}}
\newcommand{\vecBas}[2]{\tensor*[]{\left[#1\right]}{_{#2}}}

% Topology
\newcommand{\closure}[1]{\bar{#1}}

% Proofs
\newcommand{\st}{s.t.}
\newcommand{\unique}{!}

%%%%%%%%%%%%%%%%%%%%%%%%%%%%%
% Other commands
%%%%%%%%%%%%%%%%%%%%%%%%%%%%%
\newcommand{\flag}[1]{\textbf{\textcolor{red}{#1}}}

%%%%%%%%%%%%%%%%%%%%%%%%%%%%%
% Make l's curvy in math environments
%%%%%%%%%%%%%%%%%%%%%%%%%%%%%
\mathcode`l="8000
\begingroup
\makeatletter
\lccode`\~=`\l
\DeclareMathSymbol{\lsb@l}{\mathalpha}{letters}{`l}
\lowercase{\gdef~{\ifnum\the\mathgroup=\m@ne \ell \else \lsb@l \fi}}%
\endgroup

\author{Dr. Danny Nguyen\\ \small\textit{Transcribed by Thomas Cohn}}
\title{Matrix-Tree Theorem}
\date{9/11/18} % Can also use \today

\begin{document}
\maketitle
\setlength\RaggedRightParindent{\parindent}
\RaggedRight

\defn{Given $G=(V,E)$, the \underline{adjacency matrix} $A=(a_{ij})$ with $\displaystyle{}a_{ij}=\left\{\begin{array}{ll}1 & (v_{i},v_{j})\in{}E\\ 0 & \ptxt{otherwise}\end{array}\right.$}

\defn{The \underline{laplacian} $L=(l_{ij})$ is defined by $\displaystyle{}l_{ij}=\left\{\begin{array}{ll}\deg{v_{i}} & i=j\\ -a_{ij} & i\ne{}j\end{array}\right.$}

\par\noindent Observe: $\det(L)=0$. This is because $L\cdot\left(\begin{array}{c}1\\ \vdots\\ 1\end{array}\right)=\left(\begin{array}{c}0\\ \vdots\\ 0\end{array}\right)$, so $\ker{L}\ne{}0$.\n

\defn{For each $1\le{}i\le{}n$, the \underline{minor} $L_{i,j}$ is the determinant of $L$ where the $i$-th row and $j$-th column are deleted.}

\ex{\inlineeq{L=\left[\begin{array}{rrrr}2 & -1 & -1 & 0\\
-1 & 3 & -1 & -1\\
-1 & -1 & 3 & -1\\
0 & -1 & -1 & 2\end{array}\right]\qquad{}
L_{11}=\det{\left[\begin{array}{rrr}3 & -1 & -1\\
-1 & 3 & -1\\
-1 & -1 & 2\end{array}\right]}=8\qquad{}
L_{4}=\det{\left[\begin{array}{rrr}2 & -1 & -1\\
-1 & 3 & -1\\
-1 & -1 & 3\end{array}\right]}=8}}

\par\noindent Notice that every minor of the Laplacian is the same.\n

\thm{(Matrix Tree) Given any graph $G$, the number of spanning trees in $G=L_{i,i}$ for any $1\le{}l\le{}n$. ($L$ is the Laplacian.)\n
Proof: Let $e_{1},\ldots,e_{m}$ be all the edges in $G$. Orient each edge $e_{i}$ arbitrarily (denote the oriented edge $\vec{e_{i}}$).\n
Define the incidence matrix $N\in\R^{n\times{}m}$ by $\displaystyle{}n_{ij}=\left\{\begin{array}{ll}0 & x_{i}\not\in\vec{e_{j}}\\ 1 & x_{i}=\ptxt{head}(\vec{e_{j}})\\ -1 & x_{i}=\ptxt{tail}(\vec{e_{j}})\end{array}\right.$\n
\n
Observe that $NN\tpose=L$.\n
${}\quad$Proof: $l_{i,i}=r_{i}(N)\cdot{}r_{i}(N)=\deg{v_{i}}$. For $i\ne{}j$, $l_{i,j}=r_{i}(N)\cdot{}r_{j}(N)=\left\{\begin{array}{ll}-1 & (i,j)\in{}E\\ 0 & (i,j)\not\in{}E\end{array}\right.\square$\n
\n
So $\det(L_{1,1})=\det(N_{1}N_{1}\tpose)$, where $N_{1}$ is $N$ with row $1$ deleted.}

\lemma{(Cauchy-Binet) Let $A\in\R^{l\times{}m}$, $B\in\R^{m\times{}l}$, with $l\le{}m$.\n
\inlineeq{\det(AB)=\sum_{I=\set{i_{1},\ldots,i_{l}}\subset{}[m]}\det(A^{I})\det(B^{I})}\n
where $A^{I}$ is $A$ with cols $i_{1},\ldots,i_{l}\in{}I$ and $B^{I}$ is $B$ with rows $i_{1},\ldots,i_{l}\in{}I$.\n
This is proved in the textbook.\n}

\par\noindent\settowidth{\hangindent}{\textbf{Thm: }}\phantom{\textbf{Thm: }}So we now have $\displaystyle\det(L_{1,1})=\det(N_{1}N_{1}\tpose)=\sum_{I=\set{i_{1},\ldots,i_{n-1}}\subset{}[m]}\det(N_{1}^{I})\det((N_{1}^{I})\tpose)=\sum_{I=\set{i_{1},\ldots,i_{n-1}}\subset{}[m]}\det(N_{1}^{I})^{2}$.\n

\lemma{For each $I=\set{i_{1},\ldots,i_{n-1}}\subset{}[m]$, $\det(N_{1}^{I})=\left\{\begin{array}{ll}0 & \set{e_{i_{1}},\ldots,e_{i_{n-1}}}\ptxt{ is not a tree}\\ \pm{}1 & \set{e_{i_{1}},\ldots,e_{i_{n-1}}}\ptxt{ is a tree}\end{array}\right.$\n
Proof: If $\set{e_{i_{1}},\ldots,e_{i_{n-1}}}$ is disconnected, then there is a cycle in one of the connected components. Adding up the columns for that cycle and multiplying by $\pm{}1$ as needed gives us a column which is $\vec{0}$. So the determinant of the matrix is $0$.\n
If $\set{e_{i_{1}},\ldots,e_{i_{n-1}}}$ is a tree, then $\exists$ some leaf $v\ne{}1$ because the first row is gone. In the row of $v$, there exists a single $\pm{}1$ entry; perform Laplace Expansion along that row to calculate the determinant, and recurse.\proven}

\proven

\end{document}