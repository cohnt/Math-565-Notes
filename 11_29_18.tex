\documentclass[10pt,letterpaper]{article}
\usepackage[utf8]{inputenc}
\usepackage[intlimits]{amsmath}
\usepackage{amsfonts}
\usepackage{amssymb}
\usepackage{ragged2e}
\usepackage[letterpaper, margin=1in]{geometry}
\usepackage{graphicx}
\usepackage{cancel}
\usepackage{mathtools}
\usepackage{tabularx}
\usepackage{arydshln}
\usepackage{tensor}
\usepackage{array}
\usepackage{xcolor}
\usepackage[boxed]{algorithm}
\usepackage[noend]{algpseudocode}
\usepackage{listings}
\usepackage{textcomp}
\usepackage[pdf,tmpdir,singlefile]{graphviz}
\usepackage{mathrsfs}
\usepackage{bbm}
\usepackage{tikz}
\usepackage{enumitem}
\usepackage{arydshln}

%%%%%%%%%%%%%%%%%%%%%%%%%%%%%
% Formatting commands
%%%%%%%%%%%%%%%%%%%%%%%%%%%%%
\newcommand{\n}{\hfill\break}
\newcommand{\up}{\vspace{-\baselineskip}}
\newcommand{\lemma}[1]{\par\noindent\settowidth{\hangindent}{\textbf{Lemma: }}\textbf{Lemma: }#1}
\newcommand{\defn}[1]{\par\noindent\settowidth{\hangindent}{\textbf{Defn: }}\textbf{Defn: }#1\n}
\newcommand{\thm}[1]{\par\noindent\settowidth{\hangindent}{\textbf{Thm: }}\textbf{Thm: }#1\n}
\newcommand{\prop}[1]{\par\noindent\settowidth{\hangindent}{\textbf{Prop: }}\textbf{Prop: }#1\n}
\newcommand{\cor}[1]{\par\noindent\settowidth{\hangindent}{\textbf{Cor: }}\textbf{Cor: }#1\n}
\newcommand{\ex}[1]{\par\noindent\settowidth{\hangindent}{\textbf{Ex: }}\textbf{Ex: }#1\n}
\newcommand{\proven}{\;$\square$\n}
\newcommand{\problem}[1]{\par\noindent{#1}\n}
\newcommand{\problempart}[2]{\par\noindent\indent{}\settowidth{\hangindent}{\textbf{(#1)} \indent{}}\textbf{(#1)} #2\n}
\newcommand{\ptxt}[1]{\textrm{\textnormal{#1}}}
\newcommand{\inlineeq}[1]{\centerline{$\displaystyle #1$}}
\newcommand{\pageline}{\noindent\rule{\textwidth}{0.1pt}}

%%%%%%%%%%%%%%%%%%%%%%%%%%%%%
% Math commands
%%%%%%%%%%%%%%%%%%%%%%%%%%%%%
% Set Theory
\newcommand{\card}[1]{\left|#1\right|}
\newcommand{\set}[1]{\left\{#1\right\}}
\newcommand{\ps}[1]{\mathcal{P}\left(#1\right)}
\newcommand{\pfinite}[1]{\mathcal{P}^{\ptxt{finite}}\left(#1\right)}
\newcommand{\naturals}{\mathbb{N}}
\newcommand{\N}{\naturals}
\newcommand{\integers}{\mathbb{Z}}
\newcommand{\Z}{\integers}
\newcommand{\rationals}{\mathbb{Q}}
\newcommand{\Q}{\rationals}
\newcommand{\reals}{\mathbb{R}}
\newcommand{\R}{\reals}
\newcommand{\complex}{\mathbb{C}}
\newcommand{\C}{\complex}
\newcommand{\comp}{^{\complement}}
\DeclareMathOperator{\Hom}{Hom}
\newcommand{\Ind}{\mathbbm{1}}

% Graph Theory
\let\deg\relax
\DeclareMathOperator{\deg}{deg}
\newcommand{\degp}{\ptxt{deg}^{+}}
\newcommand{\degn}{\ptxt{deg}^{-}}
\newcommand{\precdot}{\mathrel{\ooalign{$\prec$\cr\hidewidth\hbox{$\cdot\mkern0.5mu$}\cr}}}
\DeclareMathOperator{\cl}{cl}

% Probability
\newcommand{\Prob}{\mathbb{P}}
\newcommand{\Avg}{\mathbb{E}}

% Standard Math
\newcommand{\inv}{^{-1}}
\newcommand{\abs}[1]{\left|#1\right|}
\newcommand{\ceil}[1]{\left\lceil{}#1\right\rceil{}}
\newcommand{\floor}[1]{\left\lfloor{}#1\right\rfloor{}}
\newcommand{\conj}[1]{\overline{#1}}
\newcommand{\of}{\circ}
\newcommand{\tri}{\triangle}
\newcommand{\inj}{\hookrightarrow}
\newcommand{\surj}{\twoheadrightarrow}
\newcommand{\mapsfrom}{\mathrel{\reflectbox{\ensuremath{\mapsto}}}}
\newcommand{\ndiv}{\nmid}
\renewcommand{\epsilon}{\varepsilon}
\newcommand{\divides}{\mid}
\newcommand{\ndivdies}{\nmid}
\DeclareMathOperator{\lcm}{lcm}

% Linear Algebra
\newcommand{\Id}{\textrm{\textnormal{Id}}}
\newcommand{\im}{\textrm{\textnormal{im}}}
\newcommand{\norm}[1]{\abs{\abs{#1}}}
\newcommand{\tpose}{^{T}}
\newcommand{\iprod}[1]{\left<#1\right>}
\DeclareMathOperator{\trace}{tr}
\newcommand{\chgBasMat}[3]{\!\!\tensor*[_{#1}]{\left[#2\right]}{_{#3}}}
\newcommand{\vecBas}[2]{\tensor*[]{\left[#1\right]}{_{#2}}}
\DeclareMathOperator{\GL}{GL}
\DeclareMathOperator{\Mat}{Mat}
\DeclareMathOperator{\vspan}{span}
\DeclareMathOperator{\rank}{rank}

% Topology
\newcommand{\closure}[1]{\overline{#1}}
\newcommand{\uball}{\mathcal{U}}
\newcommand{\Int}{\ptxt{Int}\>}
\newcommand{\Ext}{\ptxt{Ext}\>}
\newcommand{\Bd}{\ptxt{Bd}\>}
\newcommand{\rInt}{\ptxt{rInt}\>}

% Analysis
\DeclareMathOperator{\Graph}{Graph}
\DeclareMathOperator{\epi}{epi}
\DeclareMathOperator{\hypo}{hypo}
\newcommand{\lint}[2]{\underset{#1}{\overset{#2}{{\color{black}\underline{{\color{white}\overline{{\color{black}\int}}\color{black}}}}}}}
\newcommand{\uint}[2]{\underset{#1}{\overset{#2}{{\color{white}\underline{{\color{black}\overline{{\color{black}\int}}\color{black}}}}}}}
\newcommand{\alignint}[2]{\underset{#1}{\overset{#2}{{\color{white}\underline{{\color{white}\overline{{\color{black}\int}}\color{black}}}}}}}

% Proofs
\newcommand{\st}{s.t.}
\newcommand{\unique}{!}

% Algorithms
\algrenewcommand{\algorithmiccomment}[1]{\hskip 1em \texttt{// #1}}
\algrenewcommand\algorithmicrequire{\textbf{Input:}}
\algrenewcommand\algorithmicensure{\textbf{Output:}}
\newcommand{\parSymbol}{\P}
\renewcommand{\P}{\ptxt{\textbf{P}}}
\newcommand{\NP}{\ptxt{\textbf{NP}}}
\newcommand{\NPC}{\ptxt{\textbf{NP-Complete}}}
\newcommand{\NPH}{\ptxt{\textbf{NP-Hard}}}
\newcommand{\EXP}{\ptxt{\textbf{EXP}}}

%%%%%%%%%%%%%%%%%%%%%%%%%%%%%
% Other commands
%%%%%%%%%%%%%%%%%%%%%%%%%%%%%
\newcommand{\flag}[1]{\textbf{\textcolor{red}{#1}}}

%%%%%%%%%%%%%%%%%%%%%%%%%%%%%
% Make l's curvy in math environments
%%%%%%%%%%%%%%%%%%%%%%%%%%%%%
\mathcode`l="8000
\begingroup
\makeatletter
\lccode`\~=`\l
\DeclareMathSymbol{\lsb@l}{\mathalpha}{letters}{`l}
\lowercase{\gdef~{\ifnum\the\mathgroup=\m@ne \ell \else \lsb@l \fi}}%
\endgroup

\newcommand{\B}{
    \begin{tikzpicture}
    \filldraw [fill=red, draw=black] (0, 0) rectangle (0.37, 0.45);
    \draw [line width=0.5mm, white ] (0.1,0.08) -- (0.1,0.38);
    \draw[line width=0.5mm, white ] (0.1, 0.35) .. controls (0.2, 0.35) and (0.4, 0.2625) .. (0.1, 0.225);
    \draw[line width=0.5mm, white ] (0.1, 0.225) .. controls (0.2, 0.225) and (0.4, 0.1625) .. (0.1, 0.1);
    \end{tikzpicture}
}

\author{Dr. Danny Nguyen\\ \small\textit{Transcribed by Thomas Cohn}}
\title{Matroids}
\date{11/29/18} % Can also use \today

\begin{document}
\maketitle
\setlength\RaggedRightParindent{\parindent}
\RaggedRight

\par\noindent This command may be useful:\n
\verb|\newcommand{\precdot}{\mathrel{\ooalign{$\prec$\cr\hidewidth\hbox{$\cdot\mkern0.5mu$}\cr}}}|

\par\noindent\n
From last time:
\prop{A lattice $\mathcal{L}$ can be ranked if and only if every maximal chain $\hat{0}\precdot{}x_{1}\precdot{}x_{2}\precdot{}\cdots\precdot{}x_{l}=\hat{1}$ has the same length.\n
Proof ``$\Rightarrow$'': $\rank(x_{i+1})=\rank(x_{i})+1$. So $l=\rank(x_{l})=\rank(\hat{1})$.\n
Proof ``$\Leftarrow$'': Inductive, starting with $\rank(\hat{0})=0$.\n
\proven}

\defn{A \underline{matriod} $M=(E,\mathcal{I})$ is a finite set $E$ and a family $\mathcal{I}$ of subsets of $E$ with $2$ properties:
\begin{enumerate}[label=M\arabic*),start=0]
	\item If $X\in\mathcal{I}$ and $Y\subseteq{}X$, then $Y\in\mathcal{I}$.
	\item If $X,Y\in\mathcal{I}$ and $\card{Y}>\card{X}$, then $\exists{}e\in{}Y\setminus{}X$ \st{} $X\cup\set{e}\in\mathcal{I}$.
\end{enumerate}\up\n
We call an $X\in\mathcal{I}$ an \underline{independent set} in $E$. A maximal independent set is called a \underline{base}.}

\ex{$M=(E,\mathcal{I})$ where $E=\set{1,2,\ldots,n}$ and $\mathcal{I}=\set{X\subseteq{}E:\card{X}\le{}k}$. (Uniform matroid).\n
$\card{I}=\binom{n}{0}+\binom{n}{1}+\cdots+\binom{n}{k}$.}

\ex{$M=(E,\mathcal{I})$ where $E\subseteq\R^{n}$ (finite set of $n$-dimensional vectors). $E=\set{v_{1},\ldots,v_{m}}$.\n
$\mathcal{I}=\set{X\subseteq{}E:X\ptxt{ linearly independent}}$. Then $M$ is called a linear matroid.
\begin{enumerate}[label=M\arabic*), start=0]
	\item $X$ linearly independent, $Y\subseteq{}X$ means $Y$ is also linearly independent, so $Y\in\mathcal{I}$.
	\item $X,Y$ linearly independent, $\card{X}<\card{Y}$. Then $\dim(\vspan(X))<\dim(\vspan(Y))$. So by basic linear algebra, we're done.
\end{enumerate}
\up}

\ex{$G=(V,E)$ graph, $M=(E,\mathcal{I})$. $\mathcal{I}=\set{X\subset{}E:X\ptxt{ is acyclic}}$ (or equivalently, $X$ forms a forest in $G$). This is a graphic matroid.
\begin{enumerate}[label=M\arabic*), start=0]
	\item $X\ptxt{ acyclic},Y\subseteq{}X\to{}Y\ptxt{ acyclic}$.
	\item $X,Y$ forests, $\card{X}<\card{Y}$. Recall that a forest with $n$ vertices and $l$ components has $n-l$ edges. So assume $\card{X}=a<b=\card{Y}$. Then $X$ has $n-a$ components and $Y$ has $n-b$ components. So there exists an edge $e$ in $Y$ connecting two components in $X$, so $X\cup\set{e}$ is still a forest.
\end{enumerate}
\up}

\prop{All bases in a matroid have the same size.\n
Proof: Assume $X$ and $Y$ are bases and $\card{X}<\card{Y}$. By M0, $\exists{}e\in{}Y\setminus{}X$ \st{} $X\cup{}E\in\mathcal{I}$. Thus, $X$ is not maximal, and is therefore not a base. Oops!\proven}

\defn{$M=(E,\mathcal{I})$, let $S\subseteq{}E$. Then $r(S)=\max\set{\card{X}:X\subseteq{}S\ptxt{ is an independent set}}$ is called the \underline{rank function}. In particular, $r(M)=r(E)$ is the size of any base.}

\par\noindent If $S\subseteq{}T$, then $r(S)\le{}r(T)$.\n

\lemma{Matroid $M=(E,\mathcal{I})$, $A\subseteq{}S\subseteq{}E$, $A$ is an independent set.\n
Then there is a set $B$ with $A\subseteq{}B\subseteq{}S$ such that $B$ is also an independent set and $\card{B}=r(S)$.\n
Proof: Let $C\subseteq{}S$ \st{} $C$ is independent and $\card{C}=r(S)$. If $\card{A}=\card{C}$, we're done. If $\card{A}<\card{C}$, by M1, there exists $e\in{}C\setminus{}A$ \st{} $A\cup\set{e}$ independent. We can repeat this until eventually we get $A\subset{}B$ independent with $\card{B}=\card{C}$.\proven}

\thm{(Rank semimodularity) Let $M=(E,\mathcal{I})$ be a matroid, $S,T\subseteq{}E$.\n
Then $r(S)+r(T)\ge{}r(S\cap{}T)+r(S\cup{}T)$.\n
Proof: Consider $S\cap{}T$, let $A\subseteq{}S\cap{}T$ such that $A$ is independent and $\card{A}=r(S\cap{}T)$. $A\subseteq{}T\cup{}T$. By our lemma, we can find $A\subseteq{}B\subseteq{}S\cup{}T$ such that $B$ independent, and $\card{B}=r(S\cup{}T)$.\n
Thus, $\card{A}=r(S\cap{}T)$ and $\card{B}=r(S\cup{}T)$. Let $B_{1}\subseteq{}S$ and $B_{2}\subseteq{}T$ both be independent sets. Then $r(S)\ge\card{B_{1}}$ and $r(T)\ge\card{B_{2}}$. So $r(S)+r(T)\ge\card{B_{1}}+\card{B_{2}}$.\n
Therefore, $\card{A}+\card{B}\ge{}r(S\cap{}T)+r(S\cup{}T)$.\proven}

\defn{$M=(E,\mathcal{I})$, $S\subseteq{}E$. We define the \underline{closure} of $S$ $\cl(S)=\set{x\in{}E:r(S)=r(S\cup{}x)}$.\n
We say $S$ is \underline{closed} or \underline{flat} if $\cl(S)=S$. We say $S$ is \underline{$k$-flat} if $\cl(S)=S$ and $r(S)=k$.}

\par\noindent Some properties of the closure:
\begin{itemize}
	\item $S\subseteq\cl(S)$
	\item $S\subseteq{}t\to\cl(S)\subseteq\cl(T)$
	\item $\cl(\cl(S))=\cl(S)$
\end{itemize}

\end{document}