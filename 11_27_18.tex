\documentclass[10pt,letterpaper]{article}
\usepackage[utf8]{inputenc}
\usepackage[intlimits]{amsmath}
\usepackage{amsfonts}
\usepackage{amssymb}
\usepackage{ragged2e}
\usepackage[letterpaper, margin=1in]{geometry}
\usepackage{graphicx}
\usepackage{cancel}
\usepackage{mathtools}
\usepackage{tabularx}
\usepackage{arydshln}
\usepackage{tensor}
\usepackage{array}
\usepackage{xcolor}
\usepackage[boxed]{algorithm}
\usepackage[noend]{algpseudocode}
\usepackage{listings}
\usepackage{textcomp}
\usepackage[pdf,tmpdir,singlefile]{graphviz}
\usepackage{mathrsfs}
\usepackage{bbm}
\usepackage{tikz}
\usepackage{enumitem}
\usepackage{arydshln}

%%%%%%%%%%%%%%%%%%%%%%%%%%%%%
% Formatting commands
%%%%%%%%%%%%%%%%%%%%%%%%%%%%%
\newcommand{\n}{\hfill\break}
\newcommand{\lemma}[1]{\par\noindent\settowidth{\hangindent}{\textbf{Lemma: }}\textbf{Lemma: }#1}
\newcommand{\defn}[1]{\par\noindent\settowidth{\hangindent}{\textbf{Defn: }}\textbf{Defn: }#1\n}
\newcommand{\thm}[1]{\par\noindent\settowidth{\hangindent}{\textbf{Thm: }}\textbf{Thm: }#1\n}
\newcommand{\prop}[1]{\par\noindent\settowidth{\hangindent}{\textbf{Prop: }}\textbf{Prop: }#1\n}
\newcommand{\cor}[1]{\par\noindent\settowidth{\hangindent}{\textbf{Cor: }}\textbf{Cor: }#1\n}
\newcommand{\ex}[1]{\par\noindent\settowidth{\hangindent}{\textbf{Ex: }}\textbf{Ex: }#1\n}
\newcommand{\proven}{\;$\square$\n}
\newcommand{\problem}[1]{\par\noindent{#1}\n}
\newcommand{\problempart}[2]{\par\noindent\indent{}\settowidth{\hangindent}{\textbf{(#1)} \indent{}}\textbf{(#1)} #2\n}
\newcommand{\ptxt}[1]{\textrm{\textnormal{#1}}}
\newcommand{\inlineeq}[1]{\centerline{$\displaystyle #1$}}
\newcommand{\pageline}{\noindent\rule{\textwidth}{0.1pt}}

%%%%%%%%%%%%%%%%%%%%%%%%%%%%%
% Math commands
%%%%%%%%%%%%%%%%%%%%%%%%%%%%%
% Set Theory
\newcommand{\card}[1]{\left|#1\right|}
\newcommand{\set}[1]{\left\{#1\right\}}
\newcommand{\ps}[1]{\mathcal{P}\left(#1\right)}
\newcommand{\pfinite}[1]{\mathcal{P}^{\ptxt{finite}}\left(#1\right)}
\newcommand{\naturals}{\mathbb{N}}
\newcommand{\N}{\naturals}
\newcommand{\integers}{\mathbb{Z}}
\newcommand{\Z}{\integers}
\newcommand{\rationals}{\mathbb{Q}}
\newcommand{\Q}{\rationals}
\newcommand{\reals}{\mathbb{R}}
\newcommand{\R}{\reals}
\newcommand{\complex}{\mathbb{C}}
\newcommand{\C}{\complex}
\newcommand{\comp}{^{\complement}}
\newcommand{\Hom}{\ptxt{Hom}\>}
\newcommand{\Ind}{\mathbbm{1}}

% Graph Theory
\renewcommand{\deg}{\ptxt{deg}}
\newcommand{\degp}{\ptxt{deg}^{+}}
\newcommand{\degn}{\ptxt{deg}^{-}}
\newcommand{\Prob}{\mathbb{P}}
\newcommand{\Avg}{\mathbb{E}}
\newcommand{\precdot}{\mathrel{\ooalign{$\prec$\cr\hidewidth\hbox{$\cdot\mkern0.5mu$}\cr}}}

% Standard Math
\newcommand{\inv}{^{-1}}
\newcommand{\abs}[1]{\left|#1\right|}
\newcommand{\ceil}[1]{\left\lceil{}#1\right\rceil{}}
\newcommand{\floor}[1]{\left\lfloor{}#1\right\rfloor{}}
\newcommand{\conj}[1]{\overline{#1}}
\newcommand{\of}{\circ}
\newcommand{\tri}{\triangle}
\newcommand{\inj}{\hookrightarrow}
\newcommand{\surj}{\twoheadrightarrow}
\newcommand{\mapsfrom}{\mathrel{\reflectbox{\ensuremath{\mapsto}}}}
\newcommand{\Graph}{\ptxt{Graph}\>}
\newcommand{\ndiv}{\nmid}
\renewcommand{\epsilon}{\varepsilon}
\newcommand{\divides}{\mid}
\newcommand{\ndivdies}{\nmid}
\DeclareMathOperator{\lcm}{lcm}

% Linear Algebra
\newcommand{\Id}{\textrm{\textnormal{Id}}}
\newcommand{\im}{\textrm{\textnormal{im}}}
\newcommand{\norm}[1]{\abs{\abs{#1}}}
\newcommand{\tpose}{^{T}}
\newcommand{\iprod}[1]{\left<#1\right>}
\newcommand{\trace}{\ptxt{tr}}
\newcommand{\chgBasMat}[3]{\!\!\tensor*[_{#1}]{\left[#2\right]}{_{#3}}}
\newcommand{\vecBas}[2]{\tensor*[]{\left[#1\right]}{_{#2}}}
\newcommand{\GL}{\ptxt{GL}\>}
\newcommand{\Mat}{\ptxt{Mat}\>}
\newcommand{\Span}{\ptxt{Span}\>}
\newcommand{\rank}{\ptxt{rank}\>}

% Topology
\newcommand{\closure}[1]{\overline{#1}}
\newcommand{\uball}{\mathcal{U}}
\newcommand{\Int}{\ptxt{Int}\>}
\newcommand{\Ext}{\ptxt{Ext}\>}
\newcommand{\Bd}{\ptxt{Bd}\>}
\newcommand{\rInt}{\ptxt{rInt}\>}

% Analysis
\newcommand{\graph}{\ptxt{graph}}
\newcommand{\epi}{\ptxt{epi}}
\newcommand{\epis}{\ptxt{epi}_{S}}
\newcommand{\hypo}{\ptxt{hypo}}
\newcommand{\hypos}{\ptxt{hypo}_{S}}
\newcommand{\lint}[2]{\underset{#1}{\overset{#2}{{\color{black}\underline{{\color{white}\overline{{\color{black}\int}}\color{black}}}}}}}
\newcommand{\uint}[2]{\underset{#1}{\overset{#2}{{\color{white}\underline{{\color{black}\overline{{\color{black}\int}}\color{black}}}}}}}
\newcommand{\alignint}[2]{\underset{#1}{\overset{#2}{{\color{white}\underline{{\color{white}\overline{{\color{black}\int}}\color{black}}}}}}}

% Proofs
\newcommand{\st}{s.t.}
\newcommand{\unique}{!}

% Algorithms
\algrenewcommand{\algorithmiccomment}[1]{\hskip 1em \texttt{// #1}}
\algrenewcommand\algorithmicrequire{\textbf{Input:}}
\algrenewcommand\algorithmicensure{\textbf{Output:}}
\newcommand{\parSymbol}{\P}
\renewcommand{\P}{\ptxt{\textbf{P}}}
\newcommand{\NP}{\ptxt{\textbf{NP}}}
\newcommand{\NPC}{\ptxt{\textbf{NP-Complete}}}
\newcommand{\NPH}{\ptxt{\textbf{NP-Hard}}}
\newcommand{\EXP}{\ptxt{\textbf{EXP}}}

%%%%%%%%%%%%%%%%%%%%%%%%%%%%%
% Other commands
%%%%%%%%%%%%%%%%%%%%%%%%%%%%%
\newcommand{\flag}[1]{\textbf{\textcolor{red}{#1}}}

%%%%%%%%%%%%%%%%%%%%%%%%%%%%%
% Make l's curvy in math environments
%%%%%%%%%%%%%%%%%%%%%%%%%%%%%
\mathcode`l="8000
\begingroup
\makeatletter
\lccode`\~=`\l
\DeclareMathSymbol{\lsb@l}{\mathalpha}{letters}{`l}
\lowercase{\gdef~{\ifnum\the\mathgroup=\m@ne \ell \else \lsb@l \fi}}%
\endgroup

\newcommand{\B}{
    \begin{tikzpicture}
    \filldraw [fill=red, draw=black] (0, 0) rectangle (0.37, 0.45);
    \draw [line width=0.5mm, white ] (0.1,0.08) -- (0.1,0.38);
    \draw[line width=0.5mm, white ] (0.1, 0.35) .. controls (0.2, 0.35) and (0.4, 0.2625) .. (0.1, 0.225);
    \draw[line width=0.5mm, white ] (0.1, 0.225) .. controls (0.2, 0.225) and (0.4, 0.1625) .. (0.1, 0.1);
    \end{tikzpicture}
}

\author{Thomas Cohn}
\title{Lattices}
\date{11/27/18} % Can also use \today

\begin{document}
\maketitle
\setlength\RaggedRightParindent{\parindent}
\RaggedRight

\defn{A \underline{lattice} $\mathcal{L}$ is a poset with two operations $x\vee{}y$ (join) and $x\wedge{}y$ (meet) with the following properties:
\begin{itemize}
	\item $x,y\preceq{}z\Leftrightarrow{}x\vee{}y\preceq{}z$
	\item $w\preceq{}x,y\Leftrightarrow{}w\preceq{}x\wedge{}y$
\end{itemize}\n
$x\vee{}y$ is the least upper bound for $x$ and $y$.\n
$x\wedge{}y$ is the greatest lower bound for $x$ and $y$.}

\par\noindent For example:

\digraph[scale=0.5]{n112718g1}{
	node[shape=point];
	edge[arrowsize=0.5];

	x->z [dir=back];
	x->t [dir=back];

	y->z [dir=back];
	y->t [dir=back];
}
\digraph[scale=0.5]{n112718g2}{
	node[shape=point];
	edge[arrowsize=0.5];

	a->b [dir=back];
	b->c [dir=back];

	e->d [dir=back];
	d->c [dir=back];
}

\par\noindent These are not lattices.\n

\ex{$P=(2^{[n]},\subseteq)$. Then we can define $S\vee{}T=S\cup{}T$ and $S\wedge{}T=S\cap{}T$. This is a lattice.}

\ex{$P=(\N,\mid)$. Then we can define $x\vee{}y=\lcm(x,y)$ and $x\wedge{}y=\gcd(x,y)$. This is a lattice.}

\prop{(a) $(a\wedge{}b)\wedge{}c=a\wedge{}(b\wedge{}c)$ and $(a\vee{}b)\vee{}c=a\vee{}(b\vee{}c)$\n
(b) $a\wedge{}(a\vee{}b)=a=a\vee{}(a\wedge{}b)$\n
Proof (a): $z_{1}=(a\wedge{}b)\wedge{}c$, $z_{2}=a\wedge{}(b\wedge{}c)$. We need to show $z_{1}=z_{2}$.\n
$z_{1}\preceq{}z_{2}$: $z_{1}\preceq{}a\wedge{}b\preceq{}a$, $z_{1}\preceq{}a\wedge{}b\preceq{}b$, $z_{1}\preceq{}c$, so $z_{1}\preceq{}b\wedge{}c$, so $z_{1}\preceq{}a\wedge(b\wedge{}c)=z_{2}$.\n
$z_{1}\succeq{}z_{2}$: follows similarly.\n
Therefore $z_{1}=z_{2}$.\proven
Proof (b): $a\preceq{}a\vee{}b\to{}a\wedge(a\vee{}b)=a$. $a\wedge{}b\preceq{}a\to{}a\vee(a\wedge{}b)=a$.\proven}

\par\noindent Remember, we can write $x_{1}\wedge{}x_{2}\wedge\cdots\wedge{}x_{n}$ no matter the ordering of the $x_{i}$'s. The same goes for $\vee$.\n

\prop{If $\mathcal{L}$ is a \textit{finite} lattice, then it has a unique smallest element $\hat{0}$ and a unique largest element $\hat{1}$.\n
Proof: Let $\hat{0}=\bigwedge_{x\in\mathcal{L}}x$, $\hat{1}=\bigvee_{x\in\mathcal{L}}x$. These are well defined by the previous proposition. $\hat{0}\preceq{}x$ and $\hat{1}\succeq{}x$, for all $x\in\mathcal{L}$.\proven}

\defn{We say $x$ \underline{covers} $y$, denoted $y\precdot{}x$, if $y\prec{}x$ and there is no $z$ such that $y\prec{}z\prec{}x$.}

\newpage
\prop{(a) If $z\precdot{}x,y$, then $z=x\wedge{}y$.\n
(b) If $x,y,\precdot{}z$, then $z=x\vee{}y$.\n
Proof (a): $z\precdot{}x,y\to{}z\prec{}x,y\to{}z\preceq{}x\wedge{}y$. If $z\ne{}x\wedge{}y$, then $z\prec{}x\wedge{}y\preceq{}x,y$, so $z\not\precdot{}x,y$. Oops! Therefore, $z=x\wedge{}y$.\proven
(b) follows similarly}

\defn{A finite lattice $\mathcal{L}$ is ranked if there is a function $\rank:\mathcal{L}\to\N$ such that
\begin{itemize}
	\item $\rank(\hat{0})=0$
	\item $\rank(x)=\rank(y)+1$ if $y\precdot{}x$.
\end{itemize}}

\par\noindent Not all lattices can be ranked. For example, consider\n
\digraph[scale=0.5]{n112718g3}{
	node[shape=point];
	edge[arrowsize=0.5];

	a->b;
	b->c;

	a->d;
	d->e;
	e->c;
}

\prop{$\mathcal{L}$ can be ranked if and only if all maximal chains $\hat{0}\precdot{}x_{1}\precdot{}x_{2}\precdot\cdots\precdot\hat{1}$ have the same length.\n
Proof: Assume $\mathcal{L}$ is ranked. Then a given chain must have a length of $\rank(\hat{1})$. So all chains must have a length of $\rank(\hat{1})$.}

\end{document}