\documentclass[10pt,letterpaper]{article}
\usepackage[utf8]{inputenc}
\usepackage{amsmath}
\usepackage{amsfonts}
\usepackage{amssymb}
\usepackage{ragged2e}
\usepackage[letterpaper, margin=1in]{geometry}
\usepackage{graphicx}
\usepackage{cancel}
\usepackage{mathtools}
\usepackage{tabularx}
\usepackage{arydshln}
\usepackage{tensor}
\usepackage{array}
\usepackage{xcolor}
\usepackage[boxed]{algorithm}
\usepackage[noend]{algpseudocode}
\usepackage{listings}
\usepackage{textcomp}
\usepackage[pdf,tmpdir,singlefile]{graphviz}

%%%%%%%%%%%%%%%%%%%%%%%%%%%%%
% Formatting commands
%%%%%%%%%%%%%%%%%%%%%%%%%%%%%
\newcommand{\n}{\hfill\break}
\newcommand{\lemma}[1]{\par\noindent\settowidth{\hangindent}{\textbf{Lemma: }}\textbf{Lemma: }#1\n}
\newcommand{\defn}[1]{\par\noindent\settowidth{\hangindent}{\textbf{Defn: }}\textbf{Defn: }#1\n}
\newcommand{\thm}[1]{\par\noindent\settowidth{\hangindent}{\textbf{Thm: }}\textbf{Thm: }#1\n}
\newcommand{\prop}[1]{\par\noindent\settowidth{\hangindent}{\textbf{Prop: }}\textbf{Prop: }#1\n}
\newcommand{\cor}[1]{\par\noindent\settowidth{\hangindent}{\textbf{Cor: }}\textbf{Cor: }#1\n}
\newcommand{\ex}[1]{\par\noindent\settowidth{\hangindent}{\textbf{Ex: }}\textbf{Ex: }#1\n}
\newcommand{\proven}{\;$\square$\n}
\newcommand{\problem}[1]{\par\noindent{#1}\n}
\newcommand{\problempart}[2]{\par\noindent\indent{}\settowidth{\hangindent}{\textbf{(#1)} \indent{}}\textbf{(#1)} #2\n}
\newcommand{\ptxt}[1]{\textrm{\textnormal{#1}}}
\newcommand{\inlineeq}[1]{\centerline{$\displaystyle #1$}}
\newcommand{\pageline}{\noindent\rule{\textwidth}{0.1pt}}

%%%%%%%%%%%%%%%%%%%%%%%%%%%%%
% Math commands
%%%%%%%%%%%%%%%%%%%%%%%%%%%%%
% Set Theory
\newcommand{\card}[1]{\left|#1\right|}
\newcommand{\set}[1]{\left\{#1\right\}}
\newcommand{\ps}[1]{\mathcal{P}\left(#1\right)}
\newcommand{\pfinite}[1]{\mathcal{P}^{\ptxt{finite}}\left(#1\right)}
\newcommand{\naturals}{\mathbb{N}}
\newcommand{\N}{\naturals}
\newcommand{\integers}{\mathbb{Z}}
\newcommand{\Z}{\integers}
\newcommand{\rationals}{\mathbb{Q}}
\newcommand{\Q}{\rationals}
\newcommand{\reals}{\mathbb{R}}
\newcommand{\R}{\reals}
\newcommand{\complex}{\mathbb{C}}
\newcommand{\C}{\complex}
\newcommand{\comp}{^{\complement}}
\newcommand{\Hom}{\ptxt{Hom}\>}

% Graph Theory
\renewcommand{\deg}[1]{\ptxt{deg}\left(#1\right)}
\newcommand{\degp}[1]{\ptxt{deg}^{+}\!\!\left(#1\right)}
\newcommand{\degn}[1]{\ptxt{deg}^{-}\!\!\left(#1\right)}
\newcommand{\Prob}{\mathbb{P}}
\newcommand{\Avg}[1]{\mathbb{E}\left[#1\right]}

% Standard Math
\newcommand{\inv}{^{-1}}
\newcommand{\abs}[1]{\left|#1\right|}
\newcommand{\ceil}[1]{\left\lceil{}#1\right\rceil}
\newcommand{\floor}[1]{\left\lfloor{}#1\right\rfloor{}}
\newcommand{\conj}[1]{\overline{#1}}
\newcommand{\of}{\circ}
\newcommand{\tri}{\triangle}
\newcommand{\inj}{\hookrightarrow}
\newcommand{\surj}{\twoheadrightarrow}
\newcommand{\mapsfrom}{\mathrel{\reflectbox{\ensuremath{\mapsto}}}}
\newcommand{\Graph}{\ptxt{Graph}\>}
\newcommand{\ndiv}{\nmid}
\renewcommand{\epsilon}{\varepsilon}

% Linear Algebra
\newcommand{\Id}{\textrm{\textnormal{Id}}}
\newcommand{\im}{\textrm{\textnormal{im}}}
\newcommand{\norm}[1]{\abs{\abs{#1}}}
\newcommand{\tpose}{^{T}}
\newcommand{\iprod}[1]{\left<#1\right>}
\newcommand{\trace}{\ptxt{tr}}
\newcommand{\chgBasMat}[3]{\!\!\tensor*[_{#1}]{\left[#2\right]}{_{#3}}}
\newcommand{\vecBas}[2]{\tensor*[]{\left[#1\right]}{_{#2}}}
\newcommand{\GL}{\ptxt{GL}\>}
\newcommand{\Mat}{\ptxt{Mat}\>}

% Topology
\newcommand{\closure}[1]{\overline{#1}}
\newcommand{\uball}{\mathcal{U}}
\newcommand{\Int}{\ptxt{Int}\>}
\newcommand{\Ext}{\ptxt{Ext}\>}
\newcommand{\Bd}{\ptxt{Bd}\>}

% Proofs
\newcommand{\st}{s.t.}
\newcommand{\unique}{!}

% Algorithms
\algrenewcommand{\algorithmiccomment}[1]{\hskip 1em \texttt{// #1}}
\algrenewcommand\algorithmicrequire{\textbf{Input:}}
\algrenewcommand\algorithmicensure{\textbf{Output:}}

%%%%%%%%%%%%%%%%%%%%%%%%%%%%%
% Other commands
%%%%%%%%%%%%%%%%%%%%%%%%%%%%%
\newcommand{\flag}[1]{\textbf{\textcolor{red}{#1}}}

%%%%%%%%%%%%%%%%%%%%%%%%%%%%%
% Make l's curvy in math environments
%%%%%%%%%%%%%%%%%%%%%%%%%%%%%
\mathcode`l="8000
\begingroup
\makeatletter
\lccode`\~=`\l
\DeclareMathSymbol{\lsb@l}{\mathalpha}{letters}{`l}
\lowercase{\gdef~{\ifnum\the\mathgroup=\m@ne \ell \else \lsb@l \fi}}%
\endgroup

\author{Dr. Danny Nguyen\\ \small\textit{Transcribed by Thomas Cohn}}
\title{Probabilistic Combinatorics}
\date{10/23/18} % Can also use \today

\begin{document}
\maketitle
\setlength\RaggedRightParindent{\parindent}
\RaggedRight

\defn{A \underline{discrete random variable} $X$ takes a finite number of possible values $a_{1},\ldots,a_{n}$, with\n
\inlineeq{\Prob(X=a_{i})=p_{i}}
where $0\le{}p_{i}\le{}1$ and $\sum{}p_{i}=1$.}

\defn{The \underline{average} of $X$ is\n
\inlineeq{\Avg{X}=\sum_{i=1}^{n}a_{i}\cdot\Prob(X=a_{i})}}

\defn{$2$ random variables $X$, $Y$ are \underline{independent} if\n
\inlineeq{\Prob(X=a,Y=b)=\Prob(X=a)\cdot\Prob(Y=b)}}

\ex{$\set{O,O,O,A,A,A}$\n
$\Prob(X=A,Y=A)=\frac{1}{2}\cdot\frac{2}{5}\ne\Prob(X=A)\cdot\Prob(Y=A)=\frac{1}{2}\cdot\frac{1}{2}$\n
$\Rightarrow$ $X$ and $Y$ are dependent.}

\par\noindent Some properties:\n
$X$, $Y$ random variables.\n

\par\noindent $\Avg{X+Y}=\Avg{X}+\Avg{Y}$\n
$\displaystyle\Avg{X_{1}+\cdots+X_{n}}=\sum_{i=1}^{n}\Avg{X_{i}}$\n

\par\noindent If $X$ and $Y$ are independent, then\n
$\Avg{X\cdot{}Y}=\Avg{X}\cdot\Avg{Y}$.\n

\par\noindent If $M=\Avg{X}$, then $X$ must attain some value $a\le{}M$ and $b\ge{}M$.\n

\ex{$\Prob(X_{i}=1)=\frac{1}{2}$, $\Prob(X_{i}=0)=\frac{1}{2}$.\n
$X=X{1}+\cdots+X_{n}$.\n
$\displaystyle\Avg{X}=\sum_{i=1}^{n}\Avg{X_{i}}=\sum_{i=1}^{n}\left(1\cdot\frac{1}{2}+0\cdot\frac{1}{2}\right)=\frac{n}{2}$.\n
$\displaystyle\Avg{X}=\sum_{i=0}^{n}i\cdot\Prob(X=i)$.\n
\n
$\displaystyle\Prob(X=i)=\frac{\binom{n}{i}}{2^{n}}$, so $\displaystyle\Avg{X}=\sum_{i=0}^{n}i\cdot\frac{\binom{n}{i}}{2^{n}}=\frac{n}{2^{n}}\sum_{i=0}^{n}\binom{n-1}{i-1}=\frac{n}{2^{n}}\cdot{}2^{n-1}=\frac{n}{2}$.}

\newpage
\ex{Flip $n$ fair coins, compute the average number of runs (T T H H H T T would give us 3 runs).\n
\n
$n=1$: $\begin{array}{l}T\\ H\end{array}\to\mathbb{E}=1$.\n
$n=2$: $\begin{array}{l}HH\to{}1\\ HT\to{}2\\ TH\to{}2\\ TT\to{}1\end{array}\to\mathbb{E}=1.5$\n
$n=3$: $\ldots\to\mathbb{E}=2$.\n
\n
So we guess $\frac{n+1}{2}=1+\frac{n-1}{2}$.\n
\n
The number of runs $X=1+Y=1+Y_{1}+Y_{2}+\cdots+Y_{n-1}$, where $Y$ is the number of changes in between Heads and Tails.\n
$\displaystyle\Avg{X}=\Avg{1+Y}=1+\Avg{Y}=1+\sum_{i=1}^{n-1}\Avg{Y_{i}}=1+(n-1)(\frac{1}{2})=1+\frac{n-1}{2}$.}

\ex{$n$ hunters $x_{1},\ldots,x_{n}$.\n
$n$ rabbits $y_{1},\ldots,y_{n}$.\n
Each hunter has one shot, and is aiming at a specific rabbit.\n
\n
$\Avg{\#\ptxt{ surviving rabbits}}=?$ $\Avg{\ptxt{when }n=100}\sim37$\n
\n
$X_{i}=1$ when $y_{i}$ survives, $X_{i}=0$ otherwise.\n
$\Rightarrow$ $\displaystyle\Avg{X_{1}+\cdots+X_{n}}=\sum_{i=1}^{n}\Avg{X_{i}}=\sum_{i=1}^{n}\prod_{j=1}^{n}\frac{n-1}{n}=\sum_{i=1}^{n}\left(\frac{n-1}{n}\right)^{n}=n\cdot\left(1-\frac{1}{n}\right)^{n}\to\frac{n}{e}$.}

\prop{If $G$ is a graph on $n$ vertices, with $e$ edges, then $G$ contains a bipartite subraph with at least $e/2$ edges.\n
Proof: For each $v_{i}$, let $\Prob(v_{i}\in{}A)=\frac{1}{2}$, $\Prob(v_{i}\in{}B)=\frac{1}{2}$. $e=v_{i}v_{j}$.\n
$\Prob(e\ptxt{ goes between 2 parts})=\Prob(v_{i},v_{j}\ptxt{ are in different parts})=\frac{1}{2}$\n
$\displaystyle\Avg{\#\ptxt{ of edges between $2$ parts}}=\sum_{e}\frac{1}{2}=\frac{E}{2}$.\proven}

\defn{Given a graph $G$, a set of vertices $S$ is \underline{independent} if there are no edges between them.}

\defn{$\alpha(G)=\underset{S\ptxt{ independent set}}{\max}\card{S}$. $\alpha(G)$ is called the \underline{independent number}, \underline{coclique number}.}

\ex{$E=0\to\alpha(G)=n$\n
$E=\frac{n}{2}\to\alpha(G)=\frac{n}{2}$\n
$E=\binom{k}{2}\cdot\frac{n}{k}=\frac{n(k-1)}{2}\to\alpha(G)=k$}

\thm{(Tur\'an) If $G$ has $n$ vertices, $E$ edges, then $\alpha(G)\ge\frac{n^{2}}{2E+n}$.\n
Pf: Consider a random permutation of $v_{1},\ldots,v_{n}$. Let $\pi:[n]\leftrightarrow[n]$ be a random permutation.\n
Let $v\in{}S$ if $\pi(v)>\pi(w)$, $\forall{}w\in{}N(v)$. Then for each vector $v\in{}S$, $S\cap{}N(v)=\emptyset$.\n
\n
$\displaystyle\Avg{\card{S}}=\Avg{X_{1}+\cdots+X_{n}}=\sum_{i=1}^{n}\Avg{X_{i}}$. $\Avg{X_{i}}=\Prob(v_{i}\in{}S)=\frac{1}{\deg{v_{i}}+1}$.\n
\n
$\displaystyle\Avg{\card{S}}=\left(\sum_{i=1}{^n}\frac{i}{\deg{v_{i}}+1}\right)\left(\sum\deg{v_{i}}+1\right)\left(\sum\deg{v_{i}}+1\right)\inv$\n
\phantom{$\displaystyle\Avg{\card{S}}$}$\displaystyle{}\ge(1+\cdots+1)^{2}\cdot\left(\sum\deg{v_{i}}+1\right)\inv=\frac{n^{2}}{\sum\deg{v_{i}}+1}=\frac{n^{2}}{2\card{E}+n}$.\proven}

\end{document}