\documentclass[10pt,letterpaper]{article}
\usepackage[utf8]{inputenc}
\usepackage{amsmath}
\usepackage{amsfonts}
\usepackage{amssymb}
\usepackage{ragged2e}
\usepackage[letterpaper, margin=1in]{geometry}
\usepackage{graphicx}
\usepackage{cancel}
\usepackage{mathtools}
\usepackage{tabularx}
\usepackage{arydshln}
\usepackage{tensor}
\usepackage{array}
\usepackage{xcolor}
\usepackage[boxed]{algorithm}
\usepackage[noend]{algpseudocode}
\usepackage{listings}

%%%%%%%%%%%%%%%%%%%%%%%%%%%%%
% Formatting commands
%%%%%%%%%%%%%%%%%%%%%%%%%%%%%
\newcommand{\n}{\hfill\break}
\newcommand{\lemma}[1]{\par\noindent\settowidth{\hangindent}{\textbf{Lemma: }}\textbf{Lemma: }#1\n}
\newcommand{\defn}[1]{\par\noindent\settowidth{\hangindent}{\textbf{Defn: }}\textbf{Defn: }#1\n}
\newcommand{\thm}[1]{\par\noindent\settowidth{\hangindent}{\textbf{Thm: }}\textbf{Thm: }#1\n}
\newcommand{\prop}[1]{\par\noindent\settowidth{\hangindent}{\textbf{Prop: }}\textbf{Prop: }#1\n}
\newcommand{\cor}[1]{\par\noindent\settowidth{\hangindent}{\textbf{Cor: }}\textbf{Cor: }#1\n}
\newcommand{\ex}[1]{\par\noindent\settowidth{\hangindent}{\textbf{Ex: }}\textbf{Ex: }#1\n}
\newcommand{\proven}{\;$\square$\n}
\newcommand{\problem}[1]{\par\noindent{#1}\n}
\newcommand{\problempart}[2]{\par\settowidth{\hangindent}{\textbf{(#1)} \indent{}}\textbf{(#1)} #2\n}
\newcommand{\ptxt}[1]{\textrm{\textnormal{#1}}}
\newcommand{\inlineeq}[1]{\n\centerline{$\displaystyle #1$}}
\newcommand{\pageline}{\noindent\rule{\textwidth}{0.1pt}}

%%%%%%%%%%%%%%%%%%%%%%%%%%%%%
% Math commands
%%%%%%%%%%%%%%%%%%%%%%%%%%%%%
% Set Theory
\newcommand{\card}[1]{\left|#1\right|}
\newcommand{\set}[1]{\left\{#1\right\}}
\newcommand{\ps}[1]{\mathcal{P}\left(#1\right)}
\newcommand{\pfinite}[1]{\mathcal{P}^{\ptxt{finite}}\left(#1\right)}
\newcommand{\naturals}{\mathbb{N}}
\newcommand{\N}{\naturals}
\newcommand{\integers}{\mathbb{Z}}
\newcommand{\Z}{\integers}
\newcommand{\rationals}{\mathbb{Q}}
\newcommand{\Q}{\rationals}
\newcommand{\reals}{\mathbb{R}}
\newcommand{\R}{\reals}
\newcommand{\complex}{\mathbb{C}}
\newcommand{\C}{\complex}
\newcommand{\comp}{^{\complement}}
\newcommand{\Hom}{\ptxt{Hom}\>}

% Graph Theory
\renewcommand{\deg}[1]{\ptxt{deg}\left(#1\right)}
\newcommand{\degp}[1]{\ptxt{deg}^{+}\!\!\left(#1\right)}
\newcommand{\degn}[1]{\ptxt{deg}^{-}\!\!\left(#1\right)}

% Standard Math
\newcommand{\inv}{^{-1}}
\newcommand{\abs}[1]{\left|#1\right|}
\newcommand{\ceil}[1]{\left\lceil{}#1\right\rceil}
\newcommand{\floor}[1]{\left\lfloor{}#1\right\rfloor{}}
\newcommand{\conj}[1]{\overline{#1}}
\newcommand{\of}{\circ}
\newcommand{\tri}{\triangle}
\newcommand{\inj}{\hookrightarrow}
\newcommand{\surj}{\twoheadrightarrow}
\newcommand{\mapsfrom}{\mathrel{\reflectbox{\ensuremath{\mapsto}}}}
\newcommand{\Graph}{\ptxt{Graph}\>}

% Linear Algebra
\newcommand{\Id}{\textrm{\textnormal{Id}}}
\newcommand{\im}{\textrm{\textnormal{im}}}
\newcommand{\norm}[1]{\abs{\abs{#1}}}
\newcommand{\tpose}{^{T}}
\newcommand{\iprod}[1]{\left<#1\right>}
\newcommand{\trace}{\ptxt{tr}}
\newcommand{\chgBasMat}[3]{\!\!\tensor*[_{#1}]{\left[#2\right]}{_{#3}}}
\newcommand{\vecBas}[2]{\tensor*[]{\left[#1\right]}{_{#2}}}
\newcommand{\GL}{\ptxt{GL}\>}

% Topology
\newcommand{\closure}[1]{\bar{#1}}
\newcommand{\uball}{\mathcal{U}}
\newcommand{\Int}{\ptxt{Int}\>}
\newcommand{\Ext}{\ptxt{Ext}\>}
\newcommand{\Bd}{\ptxt{Bd}\>}

% Proofs
\newcommand{\st}{s.t.}
\newcommand{\unique}{!}

% Algorithms
\algrenewcommand{\algorithmiccomment}[1]{\hskip 1em \texttt{// #1}}
\algrenewcommand\algorithmicrequire{\textbf{Input:}}
\algrenewcommand\algorithmicensure{\textbf{Output:}}

%%%%%%%%%%%%%%%%%%%%%%%%%%%%%
% Other commands
%%%%%%%%%%%%%%%%%%%%%%%%%%%%%
\newcommand{\flag}[1]{\textbf{\textcolor{red}{#1}}}

%%%%%%%%%%%%%%%%%%%%%%%%%%%%%
% Make l's curvy in math environments
%%%%%%%%%%%%%%%%%%%%%%%%%%%%%
\mathcode`l="8000
\begingroup
\makeatletter
\lccode`\~=`\l
\DeclareMathSymbol{\lsb@l}{\mathalpha}{letters}{`l}
\lowercase{\gdef~{\ifnum\the\mathgroup=\m@ne \ell \else \lsb@l \fi}}%
\endgroup

\author{Thomas Cohn}
\title{Chromatic Polynomials}
\date{9/20/18} % Can also use \today

\begin{document}
\maketitle
\setlength\RaggedRightParindent{\parindent}
\RaggedRight

\defn{$G=(V,E)$, $k\in\N$.\n
$\chi_{G}(k)$ is the number of proper colorings of $G$ with $\le{}k$ colors.}

\ex{If $G$ has $n$ vertexes, $k$ colors, and $0$ edges, then $\chi_{G}(k)=k^{n}$.}

\ex{If $G=K_{n}$, $k$ colors, then $\chi_{G}(k)=\underbrace{k(k-1)(k-2)\cdots(k-(n-1))}_{n}$.\n
Note: If $k<n$, there is no proper coloring on $G=K_{n}$ (the pigeonhole principle can be used to prove this). Notice that $k<n\to\exists{}j\in\N$ \st{} $k+j=n$, so $\chi_{G}(k)=0$ as expected.}

\ex{$G$ a tree on $n$ vertices. Then $\chi_{G}(k)=\underbrace{k(k-1)(k-1)\cdots(k-1)}_{n-1}=k(k-1)^{n-1}$}

\par\noindent Notice that $\chi_{G}$ is a polynomial of degree $n$.\n

\thm{Give any graph $G$ on $n$ vertexes, $\chi_{G}(k)$ is a polynomial of degree $n$ in $k$.\n
$\chi_{G}(k)=c_{n}k^{n}+c_{n-1}k^{n-1}+\cdots+c_{1}k+c_{0}$, with $c_{i}\in\Z$ and $c_{n}>0,c_{n-1}\le{}0,c_{n-2}\ge{}0,\ldots$\n
Proof: Strong induction on $(n,m)=(\card{V},\card{E})$. If $m=0$, $\chi_{G}(k)=k^{n}$. If $n=1$, $\chi_{G}(k)=k$.\n
Assume the thm holds for all graphs with $<n$ vertices and all graphs on $n$ vertices with $<m$ edges.\n
Consider any edge $e=(x,y)\in{}E$.\n
Let $G_{1}=G-e$ (removing $e$); $n\to{}n$, $m\to{}m-1$.\n
Let $G_{1}=G/e$ (contracting $e$); $n\to{}n-1$, $m\to{}m'<m$. $m'=m-1-\card{N(x)\cap{}N(y)}$.\n
$\left.\begin{array}{l}G_{1}\ptxt{ has }<m\ptxt{ edges}\\ G_{2}\ptxt{ has }<n\ptxt{ vertices}\end{array}\right\}\to$ induction hypothesis holds for $G_{1}$ and $G_{2}$.\n
\n
Claim: $\chi_{G}(k)=\chi_{G_{1}}(k)-\chi_{G_{2}}(k)$. This makes sense because the number of ways to color $G_{1}$ is equal to the number of ways to color $G_{1}$ with $c(x)\ne{}c(y)$ plus the number of ways to color $G_{1}$ with $c(x)=c(y)$. And the number of ways to color $G_{2}$ is equal to the number of ways to color $G$ with $c(x)=c(y)$. So $\chi_{G_{1}}(k)-\chi_{G_{2}}(k)$ should be the number of ways to color $G$ with $c(x)\ne{}c(y)$, as intended.\n
\n
$\chi_{G_{1}}(k)=d_{n}k^{n}+d_{n-1}k^{n-1}+\cdots+d_{0}$\n
$\chi_{G_{2}}(k)=e_{n-1}k^{n-1}+e_{n-2}k^{n-2}+\cdots+e_{0}$\n
So $\chi_{G}(k)=\chi_{G_{1}}(k)-\chi_{G_{2}}(k)=d_{n}k^{n}+(d_{n-1}-e_{n-1})k^{n-1}+(d_{n-2}-e_{n-2})k^{n-2}+\cdots$\n
\n
$d_{n}>0$, so $d_{n}>0$.\n
$d_{n-1}\le{}0$, and $e_{n-1}>0$, so $(d_{n-1}-e_{n-1})\le{}0$.\n
$d_{n-2}\ge{}0$, and $e_{n-2}\le{}0$, so $(d_{n-2}-e_{n-2})\ge{}0$.\n
${}\quad\vdots$\n
\proven}

\ex{$G=C_{n}$, the cycle of size $n$. $\chi_{C_{n}}=\;?$.\n
$\chi_{C_{n}}(k)=\chi_{C_{n}-c}(k)-\chi_{C_{n}/e}(k)=\chi_{P_{n}}(k)-\chi_{C_{n-1}}(k)=k(k-1)^{n-1}-\chi_{C_{n-1}}(k)$\n
$=k(k-1)^{n-1}-k(k-1)^{n-2}+k(k-1)^{n-3}-\cdots\pm{}k(k-1)$\n
$=k((k-1)^{n-1}-(k-1)^{n-2}+(k-1)^{n-3}-\cdots\pm{}(k-1))$\n
$=k\left((k-1)\frac{1-(k-1)^{n}}{n-(k-1)}\right)=\cdots=(k-1)^{n}+(-1)^{n}(k-1)$\n}

\defn{Given $G=(V,E)$, an \underline{acyclic orientation} on $G$ is a way to orient the edges in $E$ so that we have no directed cycle.}

\defn{Given $G=(V,E)$, let $a(G)$ be the number of acyclic orientations on $G$.}

\ex{$a(K_{n})=n!$}

\end{document}