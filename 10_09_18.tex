\documentclass[10pt,letterpaper]{article}
\usepackage[utf8]{inputenc}
\usepackage{amsmath}
\usepackage{amsfonts}
\usepackage{amssymb}
\usepackage{ragged2e}
\usepackage[letterpaper, margin=1in]{geometry}
\usepackage{graphicx}
\usepackage{cancel}
\usepackage{mathtools}
\usepackage{tabularx}
\usepackage{arydshln}
\usepackage{tensor}
\usepackage{array}
\usepackage{xcolor}
\usepackage[boxed]{algorithm}
\usepackage[noend]{algpseudocode}
\usepackage{listings}
\usepackage{textcomp}
\usepackage[pdf,tmpdir,singlefile]{graphviz}

%%%%%%%%%%%%%%%%%%%%%%%%%%%%%
% Formatting commands
%%%%%%%%%%%%%%%%%%%%%%%%%%%%%
\newcommand{\n}{\hfill\break}
\newcommand{\lemma}[1]{\par\noindent\settowidth{\hangindent}{\textbf{Lemma: }}\textbf{Lemma: }#1\n}
\newcommand{\defn}[1]{\par\noindent\settowidth{\hangindent}{\textbf{Defn: }}\textbf{Defn: }#1\n}
\newcommand{\thm}[1]{\par\noindent\settowidth{\hangindent}{\textbf{Thm: }}\textbf{Thm: }#1\n}
\newcommand{\prop}[1]{\par\noindent\settowidth{\hangindent}{\textbf{Prop: }}\textbf{Prop: }#1\n}
\newcommand{\cor}[1]{\par\noindent\settowidth{\hangindent}{\textbf{Cor: }}\textbf{Cor: }#1\n}
\newcommand{\ex}[1]{\par\noindent\settowidth{\hangindent}{\textbf{Ex: }}\textbf{Ex: }#1\n}
\newcommand{\proven}{\;$\square$\n}
\newcommand{\problem}[1]{\par\noindent{#1}\n}
\newcommand{\problempart}[2]{\par\noindent\indent{}\settowidth{\hangindent}{\textbf{(#1)} \indent{}}\textbf{(#1)} #2\n}
\newcommand{\ptxt}[1]{\textrm{\textnormal{#1}}}
\newcommand{\inlineeq}[1]{\centerline{$\displaystyle #1$}}
\newcommand{\pageline}{\noindent\rule{\textwidth}{0.1pt}}

%%%%%%%%%%%%%%%%%%%%%%%%%%%%%
% Math commands
%%%%%%%%%%%%%%%%%%%%%%%%%%%%%
% Set Theory
\newcommand{\card}[1]{\left|#1\right|}
\newcommand{\set}[1]{\left\{#1\right\}}
\newcommand{\ps}[1]{\mathcal{P}\left(#1\right)}
\newcommand{\pfinite}[1]{\mathcal{P}^{\ptxt{finite}}\left(#1\right)}
\newcommand{\naturals}{\mathbb{N}}
\newcommand{\N}{\naturals}
\newcommand{\integers}{\mathbb{Z}}
\newcommand{\Z}{\integers}
\newcommand{\rationals}{\mathbb{Q}}
\newcommand{\Q}{\rationals}
\newcommand{\reals}{\mathbb{R}}
\newcommand{\R}{\reals}
\newcommand{\complex}{\mathbb{C}}
\newcommand{\C}{\complex}
\newcommand{\comp}{^{\complement}}
\newcommand{\Hom}{\ptxt{Hom}\>}

% Graph Theory
\renewcommand{\deg}[1]{\ptxt{deg}\left(#1\right)}
\newcommand{\degp}[1]{\ptxt{deg}^{+}\!\!\left(#1\right)}
\newcommand{\degn}[1]{\ptxt{deg}^{-}\!\!\left(#1\right)}

% Standard Math
\newcommand{\inv}{^{-1}}
\newcommand{\abs}[1]{\left|#1\right|}
\newcommand{\ceil}[1]{\left\lceil{}#1\right\rceil}
\newcommand{\floor}[1]{\left\lfloor{}#1\right\rfloor{}}
\newcommand{\conj}[1]{\overline{#1}}
\newcommand{\of}{\circ}
\newcommand{\tri}{\triangle}
\newcommand{\inj}{\hookrightarrow}
\newcommand{\surj}{\twoheadrightarrow}
\newcommand{\mapsfrom}{\mathrel{\reflectbox{\ensuremath{\mapsto}}}}
\newcommand{\Graph}{\ptxt{Graph}\>}
\newcommand{\ndiv}{\nmid}
\renewcommand{\epsilon}{\varepsilon}

% Linear Algebra
\newcommand{\Id}{\textrm{\textnormal{Id}}}
\newcommand{\im}{\textrm{\textnormal{im}}}
\newcommand{\norm}[1]{\abs{\abs{#1}}}
\newcommand{\tpose}{^{T}}
\newcommand{\iprod}[1]{\left<#1\right>}
\newcommand{\trace}{\ptxt{tr}}
\newcommand{\chgBasMat}[3]{\!\!\tensor*[_{#1}]{\left[#2\right]}{_{#3}}}
\newcommand{\vecBas}[2]{\tensor*[]{\left[#1\right]}{_{#2}}}
\newcommand{\GL}{\ptxt{GL}\>}
\newcommand{\Mat}{\ptxt{Mat}\>}

% Topology
\newcommand{\closure}[1]{\overline{#1}}
\newcommand{\uball}{\mathcal{U}}
\newcommand{\Int}{\ptxt{Int}\>}
\newcommand{\Ext}{\ptxt{Ext}\>}
\newcommand{\Bd}{\ptxt{Bd}\>}

% Proofs
\newcommand{\st}{s.t.}
\newcommand{\unique}{!}

% Algorithms
\algrenewcommand{\algorithmiccomment}[1]{\hskip 1em \texttt{// #1}}
\algrenewcommand\algorithmicrequire{\textbf{Input:}}
\algrenewcommand\algorithmicensure{\textbf{Output:}}

%%%%%%%%%%%%%%%%%%%%%%%%%%%%%
% Other commands
%%%%%%%%%%%%%%%%%%%%%%%%%%%%%
\newcommand{\flag}[1]{\textbf{\textcolor{red}{#1}}}

%%%%%%%%%%%%%%%%%%%%%%%%%%%%%
% Make l's curvy in math environments
%%%%%%%%%%%%%%%%%%%%%%%%%%%%%
\mathcode`l="8000
\begingroup
\makeatletter
\lccode`\~=`\l
\DeclareMathSymbol{\lsb@l}{\mathalpha}{letters}{`l}
\lowercase{\gdef~{\ifnum\the\mathgroup=\m@ne \ell \else \lsb@l \fi}}%
\endgroup

\author{Thomas Cohn}
\title{Maxflow, Mincut}
\date{10/9/18} % Can also use \today

\begin{document}
\maketitle
\setlength\RaggedRightParindent{\parindent}
\RaggedRight

\defn{Given a network $\vec{N}$, and a flow $f$, an \underline{augmenting path} is a sequence of edges $s=v_{0}\ptxt{\----}v_{1}\ptxt{\----}\cdots\ptxt{\----}v_{k}=t$ such that $f(e)>0$ if $e$ is backwards in the path and $f(e)<c(e)$ if $e$ is forwards in the path.}

\par\noindent Observe: If $f$ is a flow on $\vec{N}$, and there is an augmenting path, then $f$ is not maximal.\n

\par\noindent\textbf{Ford-Fulkerson Algorithm}\n

\thm{If there is no augmenting path, then $f$ is max.}

\defn{A \underline{cut} is a partition $V=X\sqcup{}Y$ \st{} $s\in{}X$, $t\in{}Y$.}

\lemma{(1) Given $X\sqcup{}Y$ a cut, and a flow $f$, then $\card{f}=f(X,Y)-f(Y,X)$ where $f(A,B)=\sum_{e\ptxt{ edge from }A\to{}B}f(e)$.
Proof: Let $E_{0}=\set{\ptxt{edges in }X}$, $E_{1}=\set{\ptxt{edges from }X\to{}Y}$, and $E_{2}=\set{\ptxt{edges from }Y\to{}X}$.\n
Then $RHS=\sum_{e\in{}E_{1}}f(e)-\sum_{e\in{}E_{2}}f(e)$\n
\phantom{Then $RHS$}${}=\sum_{e\in{}E_{1}}f(e)-\sum_{e\in{}E_{2}}f(e)+\sum_{e\in{}E_{0}}(f(e)-f(e))$\n
\phantom{Then $RHS$}${}=f^{-}(s)+\cancel{\sum_{x\in{}X\setminus\set{s}}f^{-}(x)-f^{+}(x)}$\n
\phantom{Then $RHS$}${}=f^{-}(s)=\card{f}$.\proven}

\defn{For a cut $X\sqcup{}Y$, its \underline{capacity} is $c(X,Y)=\sum_{e\ptxt{ edge from }X\to{}Y}c(e)$.}

\par\noindent Proof of Theorem: Assume $f$ has no augmenting paths $s\to{}t$. Define\n
$X=\set{x\in{}V:\ptxt{ there is an augmenting path }s\to{}x}$. Let $Y=V\setminus{}X$, so that $X\sqcup{}Y$ is a cut. By lemma (1), we know $\card{f}=f(X,Y)-f(Y,X)$. For any edge $e=\vec{xy}$ for $x\in{}X,y\in{}Y$, $f(e)=c(e)$. For any edge $e'=\vec{y'x'}$, $f(e)=0$.\n

\lemma{(2) Given $\vec{N}$, flow $f$ on $\vec{N}$, and a cut $X\sqcup{}Y$, $\card{f}\le{}c(X,Y)$.\n
Proof: By lemma (1), $\card{f}=f(X,Y)-f(Y,X)\le{}f(X,Y)\le{}c(X,Y)$.\proven}

\par\noindent $\card{f}=f(X,Y)-\cancel{f(Y,X)}=\sum_{e=\vec{xy}}c(e)=c(X,Y)$. If $f'$ is any other flow, by lemma (2), $\card{f'}\le{}c(X,Y)$. Therefore, $f$ is a max flow.\proven

\thm{(Maxflow-Mincut) Given $\vec{N}$, then $\underset{\ptxt{flow }f}{\max}\card{f}=\underset{\ptxt{cut }X\sqcup{}Y}{\min}(c(X,Y)$.\n
Proof: If $f$ is a flow, $X\sqcup{}Y$ is a cut, by lemma (2), $\card{f}\le{}c(X,Y)$.\n
Conversely, if $f$ is a max flow, then there is no augmenting path $s\to{}t$. By the previous argument, there is some cut $X\sqcup{}Y$ and $\card{f}=c(X,Y)$.\proven}

\newpage
\thm{(Integer Property of Flows) Given $\vec{N}$ \st{} $c(e)\in\Z_{+}$, then there is a max flow \st{} $f(e)\in\Z_{+}$.\n
Proof: Start with $f=0$. Keep repeating Ford-Fulkerson. By theorem, if $f$ is not maximal, then there is an augmenting path.\n
Ford-Fulkerson repeats at most $O(\card{E}M)$ times, where $M=\underset{e}{\max{}}c(e)\in\Z_{+}$.\proven}

\cor{This also holds for rational flows.}

\defn{Let $\vec{G}$ be a directed graph, $s,t\in\vec{G}$.\n
$a_{e}(s,t)$ is the maximum number of edge-disjoint directed paths from $s$ to $t$.\n
$a_{v}(s,t)$ is the maximum number of vertex-disjoint directed paths from $s$ to $t$.\n
$b_{e}(s,t)$ is the minimum number of edges that can be deleted to disconnect $s$ and $t$.\n
$b_{v}(s,t)$ is the minimum number of vertexes that can be deleted to disconnect $s$ and $t$.}

\thm{(Menger, directed) For any $\vec{G}$, and $s,t\in\vec{G}$, $a_{e}(s,t)=b_{e}(s,t)$ and $a_{v}(s,t)=b_{v}(s,t)$.}

\thm{(Menger, undirected) For any $G$, and $s,t\in{}G$, $a_{e}(s,t)=b_{e}(s,t)$ and $a_{v}(s,t)=b_{v}(s,t)$.}

\par\noindent Proof of directed version: Given $\vec{G}$, WOLOG we can assume $s$ has only outgoing edges and $t$ has only incoming edges. So $s$ is the source, $t$ is the sink. Let $c(e)=1$. First, we will prove that $a_{e}(s,t)=b_{e}(s,t)$.\n

\par\noindent $LHS\le{}RHS$: If there are $k$ edge-disjoint paths $s\to{}t$, ten we need to delete at least $k$ edges to disconnect each path.\n
$LHS\ge{}RHS$: We claim $LHS$ is the max flow in $\vec{G}$.\n
So $LHS=$max flow$=$min cut$=c(X,Y)=$the number of edges from $X$ to $Y\ge{}RHS$.\n
Therefore, $LHS=RHS$.

\end{document}