\documentclass[10pt,letterpaper]{article}
\usepackage[utf8]{inputenc}
\usepackage{amsmath}
\usepackage{amsfonts}
\usepackage{amssymb}
\usepackage{ragged2e}
\usepackage[letterpaper, margin=1in]{geometry}
\usepackage{graphicx}
\usepackage{cancel}
\usepackage{mathtools}
\usepackage{tabularx}
\usepackage{arydshln}
\usepackage{tensor}
\usepackage{array}
\usepackage{xcolor}
\usepackage[boxed]{algorithm}
\usepackage[noend]{algpseudocode}
\usepackage{listings}
\usepackage{textcomp}

%%%%%%%%%%%%%%%%%%%%%%%%%%%%%
% Formatting commands
%%%%%%%%%%%%%%%%%%%%%%%%%%%%%
\newcommand{\n}{\hfill\break}
\newcommand{\lemma}[1]{\par\noindent\settowidth{\hangindent}{\textbf{Lemma: }}\textbf{Lemma: }#1\n}
\newcommand{\defn}[1]{\par\noindent\settowidth{\hangindent}{\textbf{Defn: }}\textbf{Defn: }#1\n}
\newcommand{\thm}[1]{\par\noindent\settowidth{\hangindent}{\textbf{Thm: }}\textbf{Thm: }#1\n}
\newcommand{\prop}[1]{\par\noindent\settowidth{\hangindent}{\textbf{Prop: }}\textbf{Prop: }#1\n}
\newcommand{\cor}[1]{\par\noindent\settowidth{\hangindent}{\textbf{Cor: }}\textbf{Cor: }#1\n}
\newcommand{\ex}[1]{\par\noindent\settowidth{\hangindent}{\textbf{Ex: }}\textbf{Ex: }#1\n}
\newcommand{\proven}{\;$\square$\n}
\newcommand{\problem}[1]{\par\noindent{#1}\n}
\newcommand{\problempart}[2]{\par\settowidth{\hangindent}{\textbf{(#1)} \indent{}}\textbf{(#1)} #2\n}
\newcommand{\ptxt}[1]{\textrm{\textnormal{#1}}}
\newcommand{\inlineeq}[1]{\centerline{$\displaystyle #1$}}
\newcommand{\pageline}{\noindent\rule{\textwidth}{0.1pt}}

%%%%%%%%%%%%%%%%%%%%%%%%%%%%%
% Math commands
%%%%%%%%%%%%%%%%%%%%%%%%%%%%%
% Set Theory
\newcommand{\card}[1]{\left|#1\right|}
\newcommand{\set}[1]{\left\{#1\right\}}
\newcommand{\ps}[1]{\mathcal{P}\left(#1\right)}
\newcommand{\pfinite}[1]{\mathcal{P}^{\ptxt{finite}}\left(#1\right)}
\newcommand{\naturals}{\mathbb{N}}
\newcommand{\N}{\naturals}
\newcommand{\integers}{\mathbb{Z}}
\newcommand{\Z}{\integers}
\newcommand{\rationals}{\mathbb{Q}}
\newcommand{\Q}{\rationals}
\newcommand{\reals}{\mathbb{R}}
\newcommand{\R}{\reals}
\newcommand{\complex}{\mathbb{C}}
\newcommand{\C}{\complex}
\newcommand{\comp}{^{\complement}}
\newcommand{\Hom}{\ptxt{Hom}\>}

% Graph Theory
\renewcommand{\deg}[1]{\ptxt{deg}\left(#1\right)}
\newcommand{\degp}[1]{\ptxt{deg}^{+}\!\!\left(#1\right)}
\newcommand{\degn}[1]{\ptxt{deg}^{-}\!\!\left(#1\right)}

% Standard Math
\newcommand{\inv}{^{-1}}
\newcommand{\abs}[1]{\left|#1\right|}
\newcommand{\ceil}[1]{\left\lceil{}#1\right\rceil}
\newcommand{\floor}[1]{\left\lfloor{}#1\right\rfloor{}}
\newcommand{\conj}[1]{\overline{#1}}
\newcommand{\of}{\circ}
\newcommand{\tri}{\triangle}
\newcommand{\inj}{\hookrightarrow}
\newcommand{\surj}{\twoheadrightarrow}
\newcommand{\mapsfrom}{\mathrel{\reflectbox{\ensuremath{\mapsto}}}}
\newcommand{\Graph}{\ptxt{Graph}\>}
\newcommand{\ndiv}{\nmid}

% Linear Algebra
\newcommand{\Id}{\textrm{\textnormal{Id}}}
\newcommand{\im}{\textrm{\textnormal{im}}}
\newcommand{\norm}[1]{\abs{\abs{#1}}}
\newcommand{\tpose}{^{T}}
\newcommand{\iprod}[1]{\left<#1\right>}
\newcommand{\trace}{\ptxt{tr}}
\newcommand{\chgBasMat}[3]{\!\!\tensor*[_{#1}]{\left[#2\right]}{_{#3}}}
\newcommand{\vecBas}[2]{\tensor*[]{\left[#1\right]}{_{#2}}}
\newcommand{\GL}{\ptxt{GL}\>}
\newcommand{\Mat}{\ptxt{Mat}\>}

% Topology
\newcommand{\closure}[1]{\bar{#1}}
\newcommand{\uball}{\mathcal{U}}
\newcommand{\Int}{\ptxt{Int}\>}
\newcommand{\Ext}{\ptxt{Ext}\>}
\newcommand{\Bd}{\ptxt{Bd}\>}

% Proofs
\newcommand{\st}{s.t.}
\newcommand{\unique}{!}

% Algorithms
\algrenewcommand{\algorithmiccomment}[1]{\hskip 1em \texttt{// #1}}
\algrenewcommand\algorithmicrequire{\textbf{Input:}}
\algrenewcommand\algorithmicensure{\textbf{Output:}}

%%%%%%%%%%%%%%%%%%%%%%%%%%%%%
% Other commands
%%%%%%%%%%%%%%%%%%%%%%%%%%%%%
\newcommand{\flag}[1]{\textbf{\textcolor{red}{#1}}}

%%%%%%%%%%%%%%%%%%%%%%%%%%%%%
% Make l's curvy in math environments
%%%%%%%%%%%%%%%%%%%%%%%%%%%%%
\mathcode`l="8000
\begingroup
\makeatletter
\lccode`\~=`\l
\DeclareMathSymbol{\lsb@l}{\mathalpha}{letters}{`l}
\lowercase{\gdef~{\ifnum\the\mathgroup=\m@ne \ell \else \lsb@l \fi}}%
\endgroup

\author{Thomas Cohn}
\title{Planar Graphs}
\date{9/25/18} % Can also use \today

\begin{document}
\maketitle
\setlength\RaggedRightParindent{\parindent}
\RaggedRight

\defn{$G=(V,E)$ is \underline{planar} if there is a drawing of the graph in the 2D plane \st{} no $2$ edges cross each other.}

\ex{$V=[n]$ and $E=\emptyset$ is planar.}

\ex{$C_{n}$ and $P_{n}$ are planar.}

\ex{$K_{1}$, $K_{2}$, and $K_{3}$ are planar.}

\ex{$K_{4}$ is planar. The normal way we would draw it does not work, but if we draw it like a tetrahedron, it does.}

\ex{$K_{5}$ is \textit{not} planar.}

\ex{The Petersen graph is not planar. We can see this by contracting the edge connecting the outer corners to each of the corners of the inner shape, leaving us with $K_{5}$.}

\defn{Consider a drawing of a planar graph $G$ in $\R^{2}$. Remove the edges and vertices in the drawing. Then the connected components of $\R^{2}\setminus(V\cup{}E)$ are called the \underline{faces} of the drwaing. We denote the number of faces as $f$.}

\thm{(Euler) If $G$ is a connected planar graph, then in any planar drawing, we have $v-e+f=2$.\n
Proof: If $x$ is a leaf in $G$, delete $x$ and the attached edge. Then $v'=v-1$, $e'=e-1$, and $f'=f$. So $v'-e'+f'=(v-1)-(e-1)+f=2$.\n
Assume there are no leaves. Then $\forall{}x\in{}V$, $\deg(x)\ge{}2$. So it is not a tree, so $\exists$ a cycle $C$ in our graph. Assume WOLOG it is the smallest cycle -- that is, there is no smaller cycle contained inside it. Delete an arbitrary edge $d$ in the cycle. Then $v'=v$, $e'=e-1$, and $f'=f-1$. So $v'-e'+f'=v-(e-1)+(f-1)=2$.\n
Therefore, by induction,\proven}

\prop{If $G$ is planar, then $2e=\sum_{\varphi\in\set{faces}}\card{\set{\ptxt{edges bounding }\varphi}}$. This works because every edge is incident to exactly two faces (note that these two faces may actually be the same face).}

\ex{Prove that $K_{5}$ is not planar.\n
Proof by contradiction: assume $K_{5}$ is planar. We have $v=5$ and $e=10$, so $f=7$. But by our proposition, we have $2e=20=\sum_{\varphi\in\set{faces}}\card{\set{\ptxt{edges bounding }\varphi}}\ge\sum_{\varphi}e=ef=21$. So $20\ge{}21$. Oops!\proven}

\ex{$K_{3,3}$ is not planar.\n
Proof by contradiction: Assume $K_{3,3}$ is planar. We have $v=6$, $e=9$, so $f=5$. Since $K_{3,3}$ is a bipartite graph, every cycle must have an even number of edges. So every face must have at least $4$ edges.\n
But by our proposition, we have $2e=18=\sum_{\varphi\in\set{faces}}\card{\set{\ptxt{edges bounding }\varphi}}\ge\sum_{\varphi}4=4f=20$. So $18\ge{}21$. Oops!\proven}

\par\noindent Question: How many colors do we need to properly color a planar graph? We can observe that $4$ is enough.\n

\thm{$4$ colors is enough to properly color any planar graph.\n
Proof: \textit{Way} too hard for class.}

\par\noindent Instead, we will prove two claims.\n
Claim 1: Every planar graph is $6$-colorable.\n
Claim 2: Every planar graph is $5$-colorable.\n

\thm{Claim 1: Every planar graph is $6$-colorable.\n
Proof: Observe that in any simple planar graph, $\exists{}x\in{}V$ \st{} $\deg{x}\le{}5$. To prove this observation to be true, we by contradiction assume $\forall{}x_{i}\in{}V$ $\deg{x_{i}}\ge{}6$.\n
Then $2e\ge\sum_{x_{i}\in{}V}6=6v\to{}e\ge{}3v$.\n
And $2e\ge\sum_{\varphi}3=3f\to{}e\ge\frac{3}{2}f$.\n
So $\frac{1}{3}e\ge{}v$ and $\frac{2}{3}e\ge{}f$. So $-e+v+f\le{}0$. Oops!\n
Therefore, $\exists{}x\in{}V$ \st{} $\deg{x}\ge{}5$.\n
\n
Delete $x$. Then we now have $G'=G-x$. If $G'$ is $6$ colorable, then so is $G$. $G'$ is still planar, so recurse.\proven}

\par\noindent Remark: The planar graph of the icosahedron has $\deg{v}=5$ for all $v\in{}V$.\n

\thm{Claim 2: Every planar graph is $5$-colorable.\n
Proof: WOLOG, add edges to $G$ until all faces are triangles (except the big outer face). We still have $G$ planar. Adding edges cannot decrease $\chi_{G}$, so it is enough to show that $\chi_{G'}\le{}5$. We will use induction.\n
\n
Let $V=V_{b}\sqcup{}V_{i}$, with $V_{b}$ boundary vertexes and $V_{i}$ interior vertexes.\n
For any $x\in{}V_{i}$, the set of possible colors $C(x)=\set{c_{1},\ldots,c_{5}}$.\n
For any $x\in{}V_{b}$, the set of possible colors $C(x)$ follows $\card{C(x)}=3$.\n
For some adjacent $x_{1},x_{2}\in{}V_{b}$, we fix $C(x_{1})=\set{c_{1}}$ and $C(x_{2})=\set{c_{2}}$.\n
We claim $\exists{}f:V\to\set{c_{1},\ldots,c_{5}}$ \st{} $f(x_{i})\in{}C(x_{i})$ and $x_{i},x_{j}$ adjacent $\to{}f(x_{i})\ne{}f(x_{j})$.\n
\n
If $v\le{}3$ (the base case), all is good!\n
Case 1: $\exists{}x_{i},x_{j}\in{}V^{b}$ \st{} $x_{j}$ and $x_{i}$ are non-adjacent on the boundary, but adjacent in the graph.\n
Then $G=G_{1}\cup{}G_{2}$ with $G_{1}\cap{}G_{2}=\set{x_{i},x_{j}}$. So $\forall(x_{1},x_{2})\ne(x_{i},x_{j})$ we have $x_{1},x_{2}\in{}G_{1}$ or $x_{1},x_{2}\in{}G_{2}$. By induction, $G_{1}$ has a coloring. In that coloring, assume $f(x_{i})=c$ and $f(x_{j})=c'$. Then let $x_{i},x_{j}$ be two special boundary vertices for $G_{2}$, with $C(x_{i})=\set{c}$ and $C(x_{j})=\set{c'}$. So by induction, $G_{2}$ has a coloring. So we can combine $G_{1}$ and $G_{2}$ to color $G$.\n
\n
Case 2: No such $x_{i},x_{j}$ exist. Then look at $x_{2}$ (with preceeding and succeeding vertexes $x_{1}$ and $x_{3}$ on the boundary), and $u_{1},\ldots,u_{k}$ interior adjacent points.\n
\flag{We ran out of time in class here. Danny will be sending out the rest of the proof.}\n
\proven}

\end{document}