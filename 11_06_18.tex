\documentclass[10pt,letterpaper]{article}
\usepackage[utf8]{inputenc}
\usepackage{amsmath}
\usepackage{amsfonts}
\usepackage{amssymb}
\usepackage{ragged2e}
\usepackage[letterpaper, margin=1in]{geometry}
\usepackage{graphicx}
\usepackage{cancel}
\usepackage{mathtools}
\usepackage{tabularx}
\usepackage{arydshln}
\usepackage{tensor}
\usepackage{array}
\usepackage{xcolor}
\usepackage[boxed]{algorithm}
\usepackage[noend]{algpseudocode}
\usepackage{listings}
\usepackage{textcomp}
\usepackage[pdf,tmpdir,singlefile]{graphviz}
\usepackage{mathrsfs}

%%%%%%%%%%%%%%%%%%%%%%%%%%%%%
% Formatting commands
%%%%%%%%%%%%%%%%%%%%%%%%%%%%%
\newcommand{\n}{\hfill\break}
\newcommand{\lemma}[1]{\par\noindent\settowidth{\hangindent}{\textbf{Lemma: }}\textbf{Lemma: }#1}
\newcommand{\defn}[1]{\par\noindent\settowidth{\hangindent}{\textbf{Defn: }}\textbf{Defn: }#1\n}
\newcommand{\thm}[1]{\par\noindent\settowidth{\hangindent}{\textbf{Thm: }}\textbf{Thm: }#1\n}
\newcommand{\prop}[1]{\par\noindent\settowidth{\hangindent}{\textbf{Prop: }}\textbf{Prop: }#1\n}
\newcommand{\cor}[1]{\par\noindent\settowidth{\hangindent}{\textbf{Cor: }}\textbf{Cor: }#1\n}
\newcommand{\ex}[1]{\par\noindent\settowidth{\hangindent}{\textbf{Ex: }}\textbf{Ex: }#1\n}
\newcommand{\proven}{\;$\square$\n}
\newcommand{\problem}[1]{\par\noindent{#1}\n}
\newcommand{\problempart}[2]{\par\noindent\indent{}\settowidth{\hangindent}{\textbf{(#1)} \indent{}}\textbf{(#1)} #2\n}
\newcommand{\ptxt}[1]{\textrm{\textnormal{#1}}}
\newcommand{\inlineeq}[1]{\centerline{$\displaystyle #1$}}
\newcommand{\pageline}{\noindent\rule{\textwidth}{0.1pt}}

%%%%%%%%%%%%%%%%%%%%%%%%%%%%%
% Math commands
%%%%%%%%%%%%%%%%%%%%%%%%%%%%%
% Set Theory
\newcommand{\card}[1]{\left|#1\right|}
\newcommand{\set}[1]{\left\{#1\right\}}
\newcommand{\ps}[1]{\mathcal{P}\left(#1\right)}
\newcommand{\pfinite}[1]{\mathcal{P}^{\ptxt{finite}}\left(#1\right)}
\newcommand{\naturals}{\mathbb{N}}
\newcommand{\N}{\naturals}
\newcommand{\integers}{\mathbb{Z}}
\newcommand{\Z}{\integers}
\newcommand{\rationals}{\mathbb{Q}}
\newcommand{\Q}{\rationals}
\newcommand{\reals}{\mathbb{R}}
\newcommand{\R}{\reals}
\newcommand{\complex}{\mathbb{C}}
\newcommand{\C}{\complex}
\newcommand{\comp}{^{\complement}}
\newcommand{\Hom}{\ptxt{Hom}\>}

% Graph Theory
\renewcommand{\deg}[1]{\ptxt{deg}}
\newcommand{\degp}[1]{\ptxt{deg}^{+}\!\!}
\newcommand{\degn}[1]{\ptxt{deg}^{-}\!\!}
\newcommand{\Prob}{\mathbb{P}}
\newcommand{\Avg}{\mathbb{E}}

% Standard Math
\newcommand{\inv}{^{-1}}
\newcommand{\abs}[1]{\left|#1\right|}
\newcommand{\ceil}[1]{\left\lceil{}#1\right\rceil}
\newcommand{\floor}[1]{\left\lfloor{}#1\right\rfloor{}}
\newcommand{\conj}[1]{\overline{#1}}
\newcommand{\of}{\circ}
\newcommand{\tri}{\triangle}
\newcommand{\inj}{\hookrightarrow}
\newcommand{\surj}{\twoheadrightarrow}
\newcommand{\mapsfrom}{\mathrel{\reflectbox{\ensuremath{\mapsto}}}}
\newcommand{\Graph}{\ptxt{Graph}\>}
\newcommand{\ndiv}{\nmid}
\renewcommand{\epsilon}{\varepsilon}

% Linear Algebra
\newcommand{\Id}{\textrm{\textnormal{Id}}}
\newcommand{\im}{\textrm{\textnormal{im}}}
\newcommand{\norm}[1]{\abs{\abs{#1}}}
\newcommand{\tpose}{^{T}}
\newcommand{\iprod}[1]{\left<#1\right>}
\newcommand{\trace}{\ptxt{tr}}
\newcommand{\chgBasMat}[3]{\!\!\tensor*[_{#1}]{\left[#2\right]}{_{#3}}}
\newcommand{\vecBas}[2]{\tensor*[]{\left[#1\right]}{_{#2}}}
\newcommand{\GL}{\ptxt{GL}\>}
\newcommand{\Mat}{\ptxt{Mat}\>}
\newcommand{\Span}{\ptxt{Span}}
\newcommand{\rank}{\ptxt{rank}\>}

% Topology
\newcommand{\closure}[1]{\overline{#1}}
\newcommand{\uball}{\mathcal{U}}
\newcommand{\Int}{\ptxt{Int}\>}
\newcommand{\Ext}{\ptxt{Ext}\>}
\newcommand{\Bd}{\ptxt{Bd}\>}
\newcommand{\rInt}{\ptxt{rInt}\>}

% Analysis
\newcommand{\graph}{\ptxt{graph}}
\newcommand{\epi}{\ptxt{epi}}
\newcommand{\epis}{\ptxt{epi}_{S}}
\newcommand{\hypo}{\ptxt{hypo}}
\newcommand{\hypos}{\ptxt{hypo}_{S}}

% Proofs
\newcommand{\st}{s.t.}
\newcommand{\unique}{!}

% Algorithms
\algrenewcommand{\algorithmiccomment}[1]{\hskip 1em \texttt{// #1}}
\algrenewcommand\algorithmicrequire{\textbf{Input:}}
\algrenewcommand\algorithmicensure{\textbf{Output:}}
\newcommand{\parSymbol}{\P}
\renewcommand{\P}{\ptxt{\textbf{P}}}
\newcommand{\NP}{\ptxt{\textbf{NP}}}

%%%%%%%%%%%%%%%%%%%%%%%%%%%%%
% Other commands
%%%%%%%%%%%%%%%%%%%%%%%%%%%%%
\newcommand{\flag}[1]{\textbf{\textcolor{red}{#1}}}

%%%%%%%%%%%%%%%%%%%%%%%%%%%%%
% Make l's curvy in math environments
%%%%%%%%%%%%%%%%%%%%%%%%%%%%%
\mathcode`l="8000
\begingroup
\makeatletter
\lccode`\~=`\l
\DeclareMathSymbol{\lsb@l}{\mathalpha}{letters}{`l}
\lowercase{\gdef~{\ifnum\the\mathgroup=\m@ne \ell \else \lsb@l \fi}}%
\endgroup

\author{Thomas Cohn}
\title{Partially Ordered Sets}
\date{11/6/18} % Can also use \today

\begin{document}
\maketitle
\setlength\RaggedRightParindent{\parindent}
\RaggedRight

\defn{A \underline{partially ordered set} (also known as a \underline{poset}) is a pair $P=(X,\preceq)$ with $X$ a set and $\preceq$ a relation on $X$ with the following properties:
\begin{enumerate}
	\item Reflexivity: $\forall{}x\in{}X$, $x\preceq{}x$
	\item Anti-Symmetry: $x\preceq{}y\land{}y\preceq{}x\to{}x=y$
	\item Transitivity: $x\preceq{}y\land{}y\preceq{}z\to{}x\preceq{}z$
\end{enumerate}
If $x\preceq{}y$, but $x\ne{}y$, we write $x\prec{}y$.}

\ex{$(\R,\le)$\n
$(\N,\le)$\n
$([a,b],\le)$ where $[a,b]=\set{a,a+1,\ldots,b}$}

\par\noindent In these examples, every pair of numbers is comparable. But this is not required!\n

\ex{$(\set{a,b,c,d},\preceq=\set{(a,b),(a,c),(b,d),(c,d)})$}

\defn{For $P=(X,\preceq)$ and some $x,y\in{}X$, if neither $x\preceq{}y$ nor $y\preceq{}x$, then we say $x$ and $y$ are \underline{incomparable} (or \underline{independent}).}

\defn{A poset $P=(X,\preceq)$ is a total ordering if every pair $x,y\in{}X$ is comparable.}

\ex{Our first three examples of posets above are total orderings. The last one is not.}

\par\noindent Can we define a total order on $\R^{2}$? Yes!

\par\noindent Dictionary (lexicographic) ordering:\n
$(x_{1},y_{1})\preceq(x_{2},y_{2})$ if
\begin{itemize}
	\item $x_{1}<x_{2}$
	\item $x_{1}=x_{2}$ and $y_{1}\le{}y_{2}$
\end{itemize}

\par\noindent We could also use polar coordinates:\n
$(r_{1},\theta_{1})\preceq(r_{2},\theta_{2})$ if
\begin{itemize}
	\item $r_{1}<r_{2}$
	\item $r_{1}=r_{2}$ and $\theta_{1}\le\theta_{2}$
\end{itemize}

\defn{Let $P=(X,\preceq)$. An element $x\in{}X$ is \underline{minimal} if there is no $y\in{}X$ \st{} $y\prec{}x$.}

\thm{(a) If $P=(X,\preceq)$ is a finite poset, then there exists a total ordering $\le$ on $X$ which extends $\preceq$.\n
\n
(b) If $x,y\in{}X$ are incomparable, then there are $2$ total orderings $\le_{1}$ and $\le_{2}$ \st{} $x\le_{1}y$ and $y\le_{2}x$.\n
\n
Proof (a): By induction on $\card{X}=n$.\n
Base case: $\card{X}=1$ is trivial.\n
\n
Induction: Assume that for all $\card{X}\le{}n$. Then consider $\card{X}=n+1$. We claim that if $P=(X,\preceq)$ is a finite poset, there is some minimal $x\in{}X$.\n
\n
Proof of claim: Pick a random $y\in{}X$. If $y$ is minimal, we're done. Otherwise, pice $y_{1}\prec{}y$. Repeat. Since $X$ is finite, we cannot have an infinite chain, so \textit{some} $y_{k}$ is the minimal element in $P$.\n
\n
Consider $x\in{}X$ minimal. Let $P'=(X-x,\preceq)$. Then $P'$ has $n$ elements, so there is a total order $\le'$ on $P'$ which extends $\preceq$. Because $x$ is minimal, it is less than all elements in $X-x$, so we can add it into $P'$ and still have a total order.\proven
\n
Proof (b): By induction on $\card{X}=n$.\n
Base case: $\card{X}=2$ is trivial.\n
\n
Induction: Assume that the assumption is true if $\card{X}\le{}n$. Let $X$ have size $n+1$. Then $\exists{}z\in{}X$ minimal.\n
Case 1: $z\ne{}x$ and $z\ne{}y$. By induction, $P'=(X-z,\preceq)$ has $2$ total orderings $\le_{1}'$ and $\le_{2}'$ with $x\le_{1}'y$ and $y\le_{2}'x$. Complete $\le_{1}',\le_{2}'$ to $\le_{1},\le_{2}$ by putting $z$ last.\n
Case 2: $x,y$ are both minimal elements. Let $P''=(X-\set{x,y},\preceq)$. Then $P''$ has the total ordering $\le''$. Complete $\le''$ to $\le_{1}$ or $\le_{2}$ by putting $x<y$ last or $y<x$ last.\n
Case 3: $x$ is the only minimum element in $X$. This is not possible, because $x\prec{}y$.\proven}

\par\noindent Boolean posets: $P=(2^{Y},\subseteq)$, where $2^{Y}$ is the power set of $Y$ and $\subseteq$ is the subset relation.\n

\ex{$Y=\set{1}$. Then $2^{Y}=\set{\emptyset,\set{1}}$\n
$Y=\set{1,2}$. Then $2^{Y}=\set{\emptyset,\set{1},\set{2},\set{1,2}}$}

\thm{If $P=(X,\preceq)$ is any finite poset, then it can be embedded into some boolean poset.\n
Proof: For $x\in{}X$, let $S_{x}=\set{y\in{}X:y\preceq{}x}$. Consider $(2^{X},\subseteq)$. The map $x\mapsto{}S_{x}$ is an embedding of $P$ into $(2^{X},\subseteq)$.\proven}

\end{document}