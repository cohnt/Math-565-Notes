\documentclass[10pt,letterpaper]{article}
\usepackage[utf8]{inputenc}
\usepackage{amsmath}
\usepackage{amsfonts}
\usepackage{amssymb}
\usepackage{ragged2e}
\usepackage[letterpaper, margin=1in]{geometry}
\usepackage{graphicx}
\usepackage{cancel}
\usepackage{mathtools}
\usepackage{tabularx}
\usepackage{arydshln}
\usepackage{tensor}
\usepackage{array}
\usepackage{xcolor}
\usepackage[boxed]{algorithm}
\usepackage[noend]{algpseudocode}
\usepackage{listings}
\usepackage{textcomp}
\usepackage[pdf,tmpdir,singlefile]{graphviz}

%%%%%%%%%%%%%%%%%%%%%%%%%%%%%
% Formatting commands
%%%%%%%%%%%%%%%%%%%%%%%%%%%%%
\newcommand{\n}{\hfill\break}
\newcommand{\lemma}[1]{\par\noindent\settowidth{\hangindent}{\textbf{Lemma: }}\textbf{Lemma: }#1\n}
\newcommand{\defn}[1]{\par\noindent\settowidth{\hangindent}{\textbf{Defn: }}\textbf{Defn: }#1\n}
\newcommand{\thm}[1]{\par\noindent\settowidth{\hangindent}{\textbf{Thm: }}\textbf{Thm: }#1\n}
\newcommand{\prop}[1]{\par\noindent\settowidth{\hangindent}{\textbf{Prop: }}\textbf{Prop: }#1\n}
\newcommand{\cor}[1]{\par\noindent\settowidth{\hangindent}{\textbf{Cor: }}\textbf{Cor: }#1\n}
\newcommand{\ex}[1]{\par\noindent\settowidth{\hangindent}{\textbf{Ex: }}\textbf{Ex: }#1\n}
\newcommand{\proven}{\;$\square$\n}
\newcommand{\problem}[1]{\par\noindent{#1}\n}
\newcommand{\problempart}[2]{\par\noindent\indent{}\settowidth{\hangindent}{\textbf{(#1)} \indent{}}\textbf{(#1)} #2\n}
\newcommand{\ptxt}[1]{\textrm{\textnormal{#1}}}
\newcommand{\inlineeq}[1]{\centerline{$\displaystyle #1$}}
\newcommand{\pageline}{\noindent\rule{\textwidth}{0.1pt}}

%%%%%%%%%%%%%%%%%%%%%%%%%%%%%
% Math commands
%%%%%%%%%%%%%%%%%%%%%%%%%%%%%
% Set Theory
\newcommand{\card}[1]{\left|#1\right|}
\newcommand{\set}[1]{\left\{#1\right\}}
\newcommand{\ps}[1]{\mathcal{P}\left(#1\right)}
\newcommand{\pfinite}[1]{\mathcal{P}^{\ptxt{finite}}\left(#1\right)}
\newcommand{\naturals}{\mathbb{N}}
\newcommand{\N}{\naturals}
\newcommand{\integers}{\mathbb{Z}}
\newcommand{\Z}{\integers}
\newcommand{\rationals}{\mathbb{Q}}
\newcommand{\Q}{\rationals}
\newcommand{\reals}{\mathbb{R}}
\newcommand{\R}{\reals}
\newcommand{\complex}{\mathbb{C}}
\newcommand{\C}{\complex}
\newcommand{\comp}{^{\complement}}
\newcommand{\Hom}{\ptxt{Hom}\>}

% Graph Theory
\renewcommand{\deg}[1]{\ptxt{deg}\left(#1\right)}
\newcommand{\degp}[1]{\ptxt{deg}^{+}\!\!\left(#1\right)}
\newcommand{\degn}[1]{\ptxt{deg}^{-}\!\!\left(#1\right)}

% Standard Math
\newcommand{\inv}{^{-1}}
\newcommand{\abs}[1]{\left|#1\right|}
\newcommand{\ceil}[1]{\left\lceil{}#1\right\rceil}
\newcommand{\floor}[1]{\left\lfloor{}#1\right\rfloor{}}
\newcommand{\conj}[1]{\overline{#1}}
\newcommand{\of}{\circ}
\newcommand{\tri}{\triangle}
\newcommand{\inj}{\hookrightarrow}
\newcommand{\surj}{\twoheadrightarrow}
\newcommand{\mapsfrom}{\mathrel{\reflectbox{\ensuremath{\mapsto}}}}
\newcommand{\Graph}{\ptxt{Graph}\>}
\newcommand{\ndiv}{\nmid}
\renewcommand{\epsilon}{\varepsilon}

% Linear Algebra
\newcommand{\Id}{\textrm{\textnormal{Id}}}
\newcommand{\im}{\textrm{\textnormal{im}}}
\newcommand{\norm}[1]{\abs{\abs{#1}}}
\newcommand{\tpose}{^{T}}
\newcommand{\iprod}[1]{\left<#1\right>}
\newcommand{\trace}{\ptxt{tr}}
\newcommand{\chgBasMat}[3]{\!\!\tensor*[_{#1}]{\left[#2\right]}{_{#3}}}
\newcommand{\vecBas}[2]{\tensor*[]{\left[#1\right]}{_{#2}}}
\newcommand{\GL}{\ptxt{GL}\>}
\newcommand{\Mat}{\ptxt{Mat}\>}

% Topology
\newcommand{\closure}[1]{\overline{#1}}
\newcommand{\uball}{\mathcal{U}}
\newcommand{\Int}{\ptxt{Int}\>}
\newcommand{\Ext}{\ptxt{Ext}\>}
\newcommand{\Bd}{\ptxt{Bd}\>}

% Proofs
\newcommand{\st}{s.t.}
\newcommand{\unique}{!}

% Algorithms
\algrenewcommand{\algorithmiccomment}[1]{\hskip 1em \texttt{// #1}}
\algrenewcommand\algorithmicrequire{\textbf{Input:}}
\algrenewcommand\algorithmicensure{\textbf{Output:}}

%%%%%%%%%%%%%%%%%%%%%%%%%%%%%
% Other commands
%%%%%%%%%%%%%%%%%%%%%%%%%%%%%
\newcommand{\flag}[1]{\textbf{\textcolor{red}{#1}}}

%%%%%%%%%%%%%%%%%%%%%%%%%%%%%
% Make l's curvy in math environments
%%%%%%%%%%%%%%%%%%%%%%%%%%%%%
\mathcode`l="8000
\begingroup
\makeatletter
\lccode`\~=`\l
\DeclareMathSymbol{\lsb@l}{\mathalpha}{letters}{`l}
\lowercase{\gdef~{\ifnum\the\mathgroup=\m@ne \ell \else \lsb@l \fi}}%
\endgroup

\author{Dr. Danny Nguyen\\ \small\textit{Transcribed by Thomas Cohn}}
\title{Matchings}
\date{10/2/18} % Can also use \today

\begin{document}
\maketitle
\setlength\RaggedRightParindent{\parindent}
\RaggedRight

\defn{Let $G=(V_{1}\sqcup{}V_{2},E)$ be a bipartite graph. A \underline{matching} $M\subset{}E$ is any collection of vertex-disjoint egdes. $M$ is a perfect matching iff $\card{M}=\card{V_{1}}=\card{V_{2}}$.}

\thm{(Halls) Let $S_{1},\ldots,S_{m}\subset[n]$. There is a way to pick $x_{i}\in{}S_{i}$ \st{} $x_{i}\ne{}x_{j}$, $\forall{}i,j$, $\Leftrightarrow$ $\forall$ subcollections $S_{i_{1}},\ldots,S_{i_{k}}$, we have $\card{S_{i_{1}}\cup\cdots\cup{}S_{i_{k}}}\ge{}k$ ($k\le{}m$).}

\cor{$G=(V_{1}\cup{}V_{2},E)$ with $\card{V_{1}}=\card{V_{2}}=n$ has a perfect matching $\Leftrightarrow$ $\card{N(x)}\ge\card{X}$ $\forall{}X\subset{}V_{1}$.}

\par\noindent Proof (Halls):\n
Necessary: Assume we can pick $x_{i}\in{}S_{i}$. Then $\forall{}S_{i_{j}}$, $x_{i_{j}}\in{}S_{i_{j}}$.\n

\par\noindent Sufficient: Induction on $m$.\n
$m=1$: $\card{S_{1}}\ge{}1$, so pick any $x_{1}\in{}S_{1}$.\n
$m=k$ true, let's show $m=k+1$.\n

\par\noindent Case 1: $\card{S_{i_{1}}\cup{}S_{i_{2}}\cup\cdots\cup{}S_{i_{k}}}\ge{}k$. Then we can choose any $x_{m+1}\in{}S_{m+1}$ and the statement still holds.\n
Case 2: $\exists$ some subcollection $\card{S_{i_{1}}\cup\cdots\cup{}S_{i_{k}}}=k$ for $x_{i_{1}}\in{}S_{i_{1}},\ldots,x_{i_{k}}\in{}S_{i_{k}}$. $S_{m+1}$ must have  aneighbor outside this subcollection, or otherwise, the inductive hypothesis fails.\n

\par\noindent Define $S_{j}'=S_{j}\setminus\set{S_{i_{1}},\ldots,S_{i_{k}}}$. We know $\card{S_{i_{1}}\cup\cdots\cup{}S_{i_{k}}\cup{}S_{i_{k+1}}\cup\cdots\cup{}S_{i_{k+l}}}\ge{}k+l$. So $\card{S_{i_{k+1}}'\cup\cdots\cup{}S_{i_{k+l}}'}=l+k-k=l$.\n

\par\noindent So by induction, we can pick $x_{k+1}\in{}S_{k+1}',\ldots,x_{k+l}\in{}S_{k+l}'$.\proven

\defn{A \underline{maximal} matching $M$ is a matching with $\card{M}\ge\card{M'}$ for all matchings $M'$.}

\defn{A subset $X\subset{}V$ is an \underline{edge cover} if every edge is incident to at least one vertex $x\in{}X$.}

\thm{(K\"onig) Assume $G=(V_{1}\sqcup{}V_{2},E)$ is a bipartite graph, perhaps without a perfect matching. Let $M\subset{}E$ be a maximal matching, and $C\subset{}V$ be a minimal edge cover. Then $\card{M}=\card{C}$.\n
Proof:\n
$\card{M}\le\card{C}$: This is easily true; if our matching has size $\card{C}+1$ or larger, then by the pigeonhole principle, two edges hit the same vertex.\n
$\card{M}\ge\card{C}$: $\forall{}X\subset{}X_{1}$, $\card{N(x)\setminus{}Y_{1}}\ge\card{X}$. If not, we can decrease the size of the edge cover. \flag{Sort of a proof by picture.} So for each element in $C$, pick a match.\n
\proven}

\end{document}