\documentclass[10pt,letterpaper]{article}
\usepackage[utf8]{inputenc}
\usepackage{amsmath}
\usepackage{amsfonts}
\usepackage{amssymb}
\usepackage{ragged2e}
\usepackage[letterpaper, margin=1in]{geometry}
\usepackage{graphicx}
\usepackage{cancel}
\usepackage{mathtools}
\usepackage{tabularx}
\usepackage{arydshln}
\usepackage{tensor}
\usepackage{array}
\usepackage{xcolor}
\usepackage[boxed]{algorithm}
\usepackage[noend]{algpseudocode}
\usepackage{listings}
\usepackage{textcomp}
\usepackage[pdf,tmpdir,singlefile]{graphviz}

%%%%%%%%%%%%%%%%%%%%%%%%%%%%%
% Formatting commands
%%%%%%%%%%%%%%%%%%%%%%%%%%%%%
\newcommand{\n}{\hfill\break}
\newcommand{\lemma}[1]{\par\noindent\settowidth{\hangindent}{\textbf{Lemma: }}\textbf{Lemma: }#1\n}
\newcommand{\defn}[1]{\par\noindent\settowidth{\hangindent}{\textbf{Defn: }}\textbf{Defn: }#1\n}
\newcommand{\thm}[1]{\par\noindent\settowidth{\hangindent}{\textbf{Thm: }}\textbf{Thm: }#1\n}
\newcommand{\prop}[1]{\par\noindent\settowidth{\hangindent}{\textbf{Prop: }}\textbf{Prop: }#1\n}
\newcommand{\cor}[1]{\par\noindent\settowidth{\hangindent}{\textbf{Cor: }}\textbf{Cor: }#1\n}
\newcommand{\ex}[1]{\par\noindent\settowidth{\hangindent}{\textbf{Ex: }}\textbf{Ex: }#1\n}
\newcommand{\proven}{\;$\square$\n}
\newcommand{\problem}[1]{\par\noindent{#1}\n}
\newcommand{\problempart}[2]{\par\noindent\indent{}\settowidth{\hangindent}{\textbf{(#1)} \indent{}}\textbf{(#1)} #2\n}
\newcommand{\ptxt}[1]{\textrm{\textnormal{#1}}}
\newcommand{\inlineeq}[1]{\centerline{$\displaystyle #1$}}
\newcommand{\pageline}{\noindent\rule{\textwidth}{0.1pt}}

%%%%%%%%%%%%%%%%%%%%%%%%%%%%%
% Math commands
%%%%%%%%%%%%%%%%%%%%%%%%%%%%%
% Set Theory
\newcommand{\card}[1]{\left|#1\right|}
\newcommand{\set}[1]{\left\{#1\right\}}
\newcommand{\ps}[1]{\mathcal{P}\left(#1\right)}
\newcommand{\pfinite}[1]{\mathcal{P}^{\ptxt{finite}}\left(#1\right)}
\newcommand{\naturals}{\mathbb{N}}
\newcommand{\N}{\naturals}
\newcommand{\integers}{\mathbb{Z}}
\newcommand{\Z}{\integers}
\newcommand{\rationals}{\mathbb{Q}}
\newcommand{\Q}{\rationals}
\newcommand{\reals}{\mathbb{R}}
\newcommand{\R}{\reals}
\newcommand{\complex}{\mathbb{C}}
\newcommand{\C}{\complex}
\newcommand{\comp}{^{\complement}}
\newcommand{\Hom}{\ptxt{Hom}\>}

% Graph Theory
\renewcommand{\deg}[1]{\ptxt{deg}\left(#1\right)}
\newcommand{\degp}[1]{\ptxt{deg}^{+}\!\!\left(#1\right)}
\newcommand{\degn}[1]{\ptxt{deg}^{-}\!\!\left(#1\right)}

% Standard Math
\newcommand{\inv}{^{-1}}
\newcommand{\abs}[1]{\left|#1\right|}
\newcommand{\ceil}[1]{\left\lceil{}#1\right\rceil}
\newcommand{\floor}[1]{\left\lfloor{}#1\right\rfloor{}}
\newcommand{\conj}[1]{\overline{#1}}
\newcommand{\of}{\circ}
\newcommand{\tri}{\triangle}
\newcommand{\inj}{\hookrightarrow}
\newcommand{\surj}{\twoheadrightarrow}
\newcommand{\mapsfrom}{\mathrel{\reflectbox{\ensuremath{\mapsto}}}}
\newcommand{\Graph}{\ptxt{Graph}\>}
\newcommand{\ndiv}{\nmid}
\renewcommand{\epsilon}{\varepsilon}

% Linear Algebra
\newcommand{\Id}{\textrm{\textnormal{Id}}}
\newcommand{\im}{\textrm{\textnormal{im}}}
\newcommand{\norm}[1]{\abs{\abs{#1}}}
\newcommand{\tpose}{^{T}}
\newcommand{\iprod}[1]{\left<#1\right>}
\newcommand{\trace}{\ptxt{tr}}
\newcommand{\chgBasMat}[3]{\!\!\tensor*[_{#1}]{\left[#2\right]}{_{#3}}}
\newcommand{\vecBas}[2]{\tensor*[]{\left[#1\right]}{_{#2}}}
\newcommand{\GL}{\ptxt{GL}\>}
\newcommand{\Mat}{\ptxt{Mat}\>}

% Topology
\newcommand{\closure}[1]{\overline{#1}}
\newcommand{\uball}{\mathcal{U}}
\newcommand{\Int}{\ptxt{Int}\>}
\newcommand{\Ext}{\ptxt{Ext}\>}
\newcommand{\Bd}{\ptxt{Bd}\>}

% Proofs
\newcommand{\st}{s.t.}
\newcommand{\unique}{!}

% Algorithms
\algrenewcommand{\algorithmiccomment}[1]{\hskip 1em \texttt{// #1}}
\algrenewcommand\algorithmicrequire{\textbf{Input:}}
\algrenewcommand\algorithmicensure{\textbf{Output:}}

%%%%%%%%%%%%%%%%%%%%%%%%%%%%%
% Other commands
%%%%%%%%%%%%%%%%%%%%%%%%%%%%%
\newcommand{\flag}[1]{\textbf{\textcolor{red}{#1}}}

%%%%%%%%%%%%%%%%%%%%%%%%%%%%%
% Make l's curvy in math environments
%%%%%%%%%%%%%%%%%%%%%%%%%%%%%
\mathcode`l="8000
\begingroup
\makeatletter
\lccode`\~=`\l
\DeclareMathSymbol{\lsb@l}{\mathalpha}{letters}{`l}
\lowercase{\gdef~{\ifnum\the\mathgroup=\m@ne \ell \else \lsb@l \fi}}%
\endgroup

\author{Thomas Cohn}
\title{Network Flow}
\date{10/4/18} % Can also use \today

\begin{document}
\maketitle
\setlength\RaggedRightParindent{\parindent}
\RaggedRight

\defn{A \underline{vertex cover} in $G$ is a collection of vertices such that every edges is incident to at least one vertex.}

\defn{A \underline{edge cover} in $G$ is a collection of edges so that every vertex is incident to at least one edge.}

\defn{An edge cover $F$ is \underline{minimal} if $\card{F}$ is minimal.}

\defn{An \underline{independent set} $I\subset{}V$ is a collection of vertices with no edges among them.}

\defn{An independent set is a \underline{maximum independent set} if $\card{I}$ is max.}

\thm{(K\"onig) If $G$ is bipartite with no isolated vertices, then the size of the maximum independent set equals the size of the minimal edge cover.}

\lemma{In any graph $G$ (not necessarily bipartite) with $n$ vertexes, we have $\card{I}=n-\card{X}$, where $I$ is a minimum independent set and $X$ is a minimum vertex cover.\n
Proof: If $S\subset{}V$ is an independent set, then $V\setminus{}S$ is a vertex cover, and vice-versa.\proven}

\lemma{(Gallai) If $G$ has no isolated vertices, then $\card{E'}=n-\card{M}$, where $E'$ is a minimum edge cover set and $M$ is a maximal matching.\n
Proof:\n
$\card{E'}+\card{M}\le{}n$: Take a maximal matching. No vertexes are isolated, so we can add one edge per vertex. And if the max matching has size $k$, and there are $l$ remaining vertexes (that is, $2k+l=n$), we can pick an edge cover of size $k+l$. $(k+l)+k=2k+l=n$. So $\card{E'}\le{}n-\card{M}$\n
\n
$\card{E'}+\card{M}\ge{}n$: Observe that if $F\subset{}E$ is a minimum edge cover, and $xy\in{}F$, then either $x$ or $y$ is incident to no other edges in $F$. If $F$ is a minimum edge cover, then $F$ is a disjoint union of ``stars''. Assume that $F$ is a minimum edge cover with $l$ stars. Then pick $1$ edge from each star. This gives us a matching of size $l$. So $l+(k_{1}+\cdots+k_{l})=(k_{1}+1)+\cdots+(k_{l}+1)=n$. And therefore, $\card{E'}+\card{M}\ge{}n$.\n
\n
Therefore, $\card{E'}=n-\card{M}$.\proven}

\par\noindent\textbf{Network Flow}\n

\defn{A network $\vec{N}=(V,E)$ is a directed graph with $2$ special vertices:
\begin{itemize}
	\item ``source'' $s$, which has only outgoing edges
	\item ``sink'' $t$, which has only incoming edges
\end{itemize}
and a capacity function $c:E\to\R_{+}$.}

\defn{A flow $f$ through $\vec{N}$ is a function $f:E\to\R_{+}$ \st{}
\begin{itemize}
	\item $f(e)\le{}c(e)$, $\forall{}e\in{}E$
	\item $\forall{}v\in{}V$ with $v\ne{}s,v\ne{}t$, we have $\displaystyle{}f^{+}(v)=\sum_{e\in\ptxt{In}(v)}f(e)=f^{-}(v)=\sum_{e\in\ptxt{Out}(v)}f(e)$
\end{itemize}}

\par\noindent Observe: If $f$ is a flow on $\vec{N}$, then $f^{-}(s)=f^{+}(t)$.\n
Proof: $0=\sum_{e\in{}E}(f^{+}(e)-f^{-}(e))=\sum_{v\in{}V}\left(\underbrace{\sum_{e\in\ptxt{In}(v)}f(e)}_{f^{+}(v)}-\underbrace{\sum_{e\in\ptxt{Out}(v)}f(e)}_{f^{-}(v)}\right)=-f^{-}(s)+f^{+}(t)$.\proven\n

\defn{If $f$ is a flow through $\vec{N}$, then the \underline{strength} of $f$ is $\card{f}=f^{+}(t)=f^{-}(s)$.}

\par\noindent Question: Given $\vec{N}$, can we find a flow $f$ with maximum $\card{f}$? Yeah probably. Consider an arbitrary network $\vec{N}$ with flow $f$.\n

\par\noindent Observe: If there is a directed path $s=x_{0}\to{}x_{1}\to\cdots\to{}x_{k}=t$ with $f(e)<c(e)$, $\forall{}e$ in the path, then we can increase the flow.\n

\par\noindent Observe: If there is a path $s=x_{0}\overset{?}{\ptxt{\----}}x_{1}\overset{?}{\ptxt{\----}}x_{2}\overset{?}{\ptxt{\----}}\cdots\overset{?}{\ptxt{\----}}x_{k}=t$ and $f(e)<c(e)$, $\forall{}e$ in the path which are ``forward'', and $f(e)>0$, $\forall{}e$ in the path which are ``backward'', then we can increase $f$.\n
Proof: For $e$ in the path, define $\mathcal{K}(e)=\left\{\begin{array}{ll}c(e)-f(e) & \quad{}e\ptxt{ is ``forward''}\\ f(e) & \quad{}e\ptxt{ is ``backward''}\end{array}\right.$ Note that $\mathcal{K}(e)>0$ for every $e$ in the path.\n
Define $\epsilon=\underset{e\in\ptxt{path}}{\min}\;\mathcal{K}(e)$, and define a new flow $f'(e)=\left\{\begin{array}{ll}f(e)+\epsilon & \quad{}e\ptxt{ is ``forward''}\\ f(e)-\epsilon & \quad{}e\ptxt{ is ``backward''}\\ f(e) & \quad{}e\ptxt{ not in the path}\end{array}\right.$\n
Then $\card{f'}=\card{f}+\epsilon$, and $f'$ still satisfies conservation of flow.\proven

\end{document}