\documentclass[10pt,letterpaper]{article}
\usepackage[utf8]{inputenc}
\usepackage[intlimits]{amsmath}
\usepackage{amsfonts}
\usepackage{amssymb}
\usepackage{ragged2e}
\usepackage[letterpaper, margin=1in]{geometry}
\usepackage{graphicx}
\usepackage{cancel}
\usepackage{mathtools}
\usepackage{tabularx}
\usepackage{arydshln}
\usepackage{tensor}
\usepackage{array}
\usepackage{xcolor}
\usepackage[boxed]{algorithm}
\usepackage[noend]{algpseudocode}
\usepackage{listings}
\usepackage{textcomp}
\usepackage[pdf,tmpdir,singlefile]{graphviz}
\usepackage{mathrsfs}
\usepackage{bbm}
\usepackage{tikz}
\usepackage{enumitem}
\usepackage{arydshln}

%%%%%%%%%%%%%%%%%%%%%%%%%%%%%
% Formatting commands
%%%%%%%%%%%%%%%%%%%%%%%%%%%%%
\newcommand{\n}{\hfill\break}
\newcommand{\up}{\vspace{-\baselineskip}}
\newcommand{\lemma}[1]{\par\noindent\settowidth{\hangindent}{\textbf{Lemma: }}\textbf{Lemma: }#1}
\newcommand{\defn}[1]{\par\noindent\settowidth{\hangindent}{\textbf{Defn: }}\textbf{Defn: }#1\n}
\newcommand{\thm}[1]{\par\noindent\settowidth{\hangindent}{\textbf{Thm: }}\textbf{Thm: }#1\n}
\newcommand{\prop}[1]{\par\noindent\settowidth{\hangindent}{\textbf{Prop: }}\textbf{Prop: }#1\n}
\newcommand{\cor}[1]{\par\noindent\settowidth{\hangindent}{\textbf{Cor: }}\textbf{Cor: }#1\n}
\newcommand{\ex}[1]{\par\noindent\settowidth{\hangindent}{\textbf{Ex: }}\textbf{Ex: }#1\n}
\newcommand{\proven}{\;$\square$\n}
\newcommand{\problem}[1]{\par\noindent{#1}\n}
\newcommand{\problempart}[2]{\par\noindent\indent{}\settowidth{\hangindent}{\textbf{(#1)} \indent{}}\textbf{(#1)} #2\n}
\newcommand{\ptxt}[1]{\textrm{\textnormal{#1}}}
\newcommand{\inlineeq}[1]{\centerline{$\displaystyle #1$}}
\newcommand{\pageline}{\noindent\rule{\textwidth}{0.1pt}}

%%%%%%%%%%%%%%%%%%%%%%%%%%%%%
% Math commands
%%%%%%%%%%%%%%%%%%%%%%%%%%%%%
% Set Theory
\newcommand{\card}[1]{\left|#1\right|}
\newcommand{\set}[1]{\left\{#1\right\}}
\newcommand{\ps}[1]{\mathcal{P}\left(#1\right)}
\newcommand{\pfinite}[1]{\mathcal{P}^{\ptxt{finite}}\left(#1\right)}
\newcommand{\naturals}{\mathbb{N}}
\newcommand{\N}{\naturals}
\newcommand{\integers}{\mathbb{Z}}
\newcommand{\Z}{\integers}
\newcommand{\rationals}{\mathbb{Q}}
\newcommand{\Q}{\rationals}
\newcommand{\reals}{\mathbb{R}}
\newcommand{\R}{\reals}
\newcommand{\complex}{\mathbb{C}}
\newcommand{\C}{\complex}
\newcommand{\comp}{^{\complement}}
\DeclareMathOperator{\Hom}{Hom}
\newcommand{\Ind}{\mathbbm{1}}

% Graph Theory
\let\deg\relax
\DeclareMathOperator{\deg}{deg}
\newcommand{\degp}{\ptxt{deg}^{+}}
\newcommand{\degn}{\ptxt{deg}^{-}}
\newcommand{\precdot}{\mathrel{\ooalign{$\prec$\cr\hidewidth\hbox{$\cdot\mkern0.5mu$}\cr}}}
\newcommand{\succdot}{\mathrel{\ooalign{$\cdot\mkern0.5mu$\cr\hidewidth\hbox{$\succ$}\cr\phantom{$\succ$}}}}
\DeclareMathOperator{\cl}{cl}
\DeclareMathOperator{\affdim}{affdim}

% Probability
\newcommand{\Prob}{\mathbb{P}}
\newcommand{\Avg}{\mathbb{E}}

% Standard Math
\newcommand{\inv}{^{-1}}
\newcommand{\abs}[1]{\left|#1\right|}
\newcommand{\ceil}[1]{\left\lceil{}#1\right\rceil{}}
\newcommand{\floor}[1]{\left\lfloor{}#1\right\rfloor{}}
\newcommand{\conj}[1]{\overline{#1}}
\newcommand{\of}{\circ}
\newcommand{\tri}{\triangle}
\newcommand{\inj}{\hookrightarrow}
\newcommand{\surj}{\twoheadrightarrow}
\newcommand{\mapsfrom}{\mathrel{\reflectbox{\ensuremath{\mapsto}}}}
\newcommand{\ndiv}{\nmid}
\renewcommand{\epsilon}{\varepsilon}
\newcommand{\divides}{\mid}
\newcommand{\ndivdies}{\nmid}
\DeclareMathOperator{\lcm}{lcm}

% Linear Algebra
\newcommand{\Id}{\textrm{\textnormal{Id}}}
\newcommand{\im}{\textrm{\textnormal{im}}}
\newcommand{\norm}[1]{\abs{\abs{#1}}}
\newcommand{\tpose}{^{T}}
\newcommand{\iprod}[1]{\left<#1\right>}
\DeclareMathOperator{\trace}{tr}
\newcommand{\chgBasMat}[3]{\!\!\tensor*[_{#1}]{\left[#2\right]}{_{#3}}}
\newcommand{\vecBas}[2]{\tensor*[]{\left[#1\right]}{_{#2}}}
\DeclareMathOperator{\GL}{GL}
\DeclareMathOperator{\Mat}{Mat}
\DeclareMathOperator{\vspan}{span}
\DeclareMathOperator{\rank}{rank}
\newcommand{\V}[1]{\vec{#1}}

% Topology
\newcommand{\closure}[1]{\overline{#1}}
\newcommand{\uball}{\mathcal{U}}
\DeclareMathOperator{\Int}{Int}
\DeclareMathOperator{\Ext}{Ext}
\DeclareMathOperator{\Bd}{Bd}
\DeclareMathOperator{\rInt}{rInt}
\DeclareMathOperator{\ch}{ch}
\DeclareMathOperator{\ah}{ah}

% Analysis
\DeclareMathOperator{\Graph}{Graph}
\DeclareMathOperator{\epi}{epi}
\DeclareMathOperator{\hypo}{hypo}
\DeclareMathOperator{\supp}{supp}
\newcommand{\lint}[2]{\underset{#1}{\overset{#2}{{\color{black}\underline{{\color{white}\overline{{\color{black}\int}}\color{black}}}}}}}
\newcommand{\uint}[2]{\underset{#1}{\overset{#2}{{\color{white}\underline{{\color{black}\overline{{\color{black}\int}}\color{black}}}}}}}
\newcommand{\alignint}[2]{\underset{#1}{\overset{#2}{{\color{white}\underline{{\color{white}\overline{{\color{black}\int}}\color{black}}}}}}}
\newcommand{\conv}{\ast}

% Proofs
\newcommand{\st}{s.t.}
\newcommand{\unique}{!}

% Brackets
\newcommand{\paren}[1]{\left(#1\right)}
\renewcommand{\brack}[1]{\left[#1\right]}
\renewcommand{\brace}[1]{\left\{#1\right\}}
\newcommand{\ang}[1]{\left<#1\right>}

% Algorithms
\algrenewcommand{\algorithmiccomment}[1]{\hskip 1em \texttt{// #1}}
\algrenewcommand\algorithmicrequire{\textbf{Input:}}
\algrenewcommand\algorithmicensure{\textbf{Output:}}
\newcommand{\parSymbol}{\P}
\renewcommand{\P}{\ptxt{\textbf{P}}}
\newcommand{\NP}{\ptxt{\textbf{NP}}}
\newcommand{\NPC}{\ptxt{\textbf{NP-Complete}}}
\newcommand{\NPH}{\ptxt{\textbf{NP-Hard}}}
\newcommand{\EXP}{\ptxt{\textbf{EXP}}}

%%%%%%%%%%%%%%%%%%%%%%%%%%%%%
% Other commands
%%%%%%%%%%%%%%%%%%%%%%%%%%%%%
\newcommand{\flag}[1]{\textbf{\textcolor{red}{#1}}}

%%%%%%%%%%%%%%%%%%%%%%%%%%%%%
% Make l's curvy in math environments
%%%%%%%%%%%%%%%%%%%%%%%%%%%%%
\mathcode`l="8000
\begingroup
\makeatletter
\lccode`\~=`\l
\DeclareMathSymbol{\lsb@l}{\mathalpha}{letters}{`l}
\lowercase{\gdef~{\ifnum\the\mathgroup=\m@ne \ell \else \lsb@l \fi}}%
\endgroup

\newcommand{\B}{
    \begin{tikzpicture}
    \filldraw [fill=red, draw=black] (0, 0) rectangle (0.37, 0.45);
    \draw [line width=0.5mm, white ] (0.1,0.08) -- (0.1,0.38);
    \draw[line width=0.5mm, white ] (0.1, 0.35) .. controls (0.2, 0.35) and (0.4, 0.2625) .. (0.1, 0.225);
    \draw[line width=0.5mm, white ] (0.1, 0.225) .. controls (0.2, 0.225) and (0.4, 0.1625) .. (0.1, 0.1);
    \end{tikzpicture}
}

\author{Dr. Danny Nguyen\\ \small\textit{Transcribed by Thomas Cohn}}
\title{Polytopes}
\date{12/11/18} % Can also use \today

\begin{document}
\maketitle
\setlength\RaggedRightParindent{\parindent}
\RaggedRight

\thm{(Helly's Theorem) Let $X_{1},\ldots,X_{M}\subseteq\R^{n}$ be convex sets.\n
If any $n+1$ of them intersect, then they all intersect.\n
Proof: If $m\le{}n+1$, we're done. So assume $m=n+2$. We have $X_{1},\ldots,X_{n+2}$.\n
Let $x_{i}\in{}X_{1}\cap\cdots\cap{}X_{i-1}\cap{}X_{i+1}\cap\cdots\cap{}X_{n+2}$ (skipping $X_{i}$), and consider $\set{x_{1},\ldots,x_{n+2}}$.\n
\n
Recall Radon's thm: $Y\subseteq\R^{n}$, $\card{Y}=n+2$, then we can partition $Y=S\sqcup{}T$ \st{} $\ch(S)\cap\ch(T)\ne\emptyset$. Applying this gives us $\set{x_{1},\ldots,x_{n+2}}=S\sqcup{}T$ with $\ch(S)\cap\ch(T)$ nonempty.\n
\n
Let $y\in\ch(S)\cap\ch(T)$, and let $1\le{}i\le{}n+2$. Then $X_{i}$ contains all $x$ for $j\ne{}i$, so either $S\subseteq{}X_{i}$ or $T\subseteq{}X_{i}$. So $y\in\ch(S)\subseteq{}X_{i}$ or $y\in\ch(T)\subseteq{}X_{i}$ for any $1\le{}i\le{}n+2$. So $m=n+2$ is done.\n
\n
Let $m\ge{}n+2$ be arbitrarily large, let $X_{1}'=X_{1}\cap{}X_{2}\ne\emptyset$. Replace $X_{1}$ and $X_{2}$ by $X_{1}'$; we claim that any $n+1$ of the new sets also intersect. If we take $X_{1}'$ and $n$ of $X_{3},\ldots,X_{m}$, then by the case $m=n+2$, we know that $(X_{1}'\cap\cdots)=(X_{1}\cap{}X_{2}\cap\cdots)\ne\emptyset$. So we can perform induction on $m$.\proven
}

\defn{A \underline{polytope} is a convex hull of finitely many points.}

\ex{$1$-dimensional polytopes: closed intervals}

\ex{$3$-dimensional polytopes: simplex, cube, octahedron, etc. (the platonic solids)}

\defn{Let $X\subseteq\R^{n}$. The \underline{affine dimension} of $X$ is $\affdim(X)=\left\{\begin{array}{ll}-1 & \quad{}x=\emptyset \\ \dim(\vspan\set{y-x:y\in{}X}) & \quad{}x\ne\emptyset\end{array}\right.$}

\defn{An \underline{$n$-dimensional simplex} $S\subseteq\R^{n}$ is a convex hull of $n+1$ points $\set{x_{0},\ldots,x_{n+1}}$ with $\affdim(S)=n$.}

\par\noindent If $P=\ch(X)$ is a polytope, then $P$ can always be triangulated $P=\bigcup_{i=1}^{m}\tri_{i}$, where each $\tri_{i}$ is a simplex with vertices in $X$, and $\Int\tri_{i}\cap\Int\tri_{j}=\emptyset$ for any $i\ne{}j$.

\defn{Faces:\n
$(-1)$-dim face: $\emptyset$\n
$0$-dim faces: vertices\n
$1$-dim faces: edges\n
$2$-dim faces: (traditional) faces\n
${}\qquad\vdots$\n
$n$-dim faces: the polytope itself}

\par\noindent The number of $i$-dimensional faces is $\binom{n}{i}2^{n-i}$.\n

\thm{(Euler-Poincare) If $P$ is an $n$-dimensional polytope, then $f_{0}-f_{1}+f_{2}-\cdots+(-1)^{d}f_{d}=1$.}

\ex{$P$ is an $n$-dimensional simplex, $f_{i}=\binom{n+1}{i+1}$. Then $\binom{n+1}{1}-\binom{n+1}{2}+\binom{n+1}{3}-\cdots+(-1)^{n}\binom{n+1}{n+1}=1$.\n
$P=C_{n}$ (cube) means $f_{i}=\binom{n}{i}2^{n-i}$, so $\binom{n}{0}2^{n}-\binom{n}{1}2^{n-1}+\cdots+(-1)^{n}\binom{n}{n}2^{0}=(2-1)^{n}=1$.}

\end{document}